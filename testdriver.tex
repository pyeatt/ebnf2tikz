\documentclass[10pt,letterpaper]{article}
\textwidth 6.5in
\evensidemargin 0in
\oddsidemargin 0in
\topmargin 0in
\headheight 0in
\headsep 0in

\pagestyle{empty}
\thispagestyle{empty}


\RequirePackage[T1]{fontenc}
\RequirePackage[full]{textcomp}
%\RequirePackage[lining,proportional]{fbb}% Bembo-like font
\RequirePackage[scale=.95,type1]{cabin}% sans serif in style of Gill Sans
\RequirePackage[varqu,varl]{zi4}% inconsolata typewriter
%\csdef{verbatim@font}{\ttfamily\fontsize{8}{10}\selectfont}
\RequirePackage[libertine,bigdelims]{newtxmath}
\let\Bbbk\relax
\RequirePackage{amsmath,amssymb,amsfonts}
\let\openbox\relax
\RequirePackage{amsthm}

\RequirePackage{rotating,graphicx}
\usepackage{tikz,pgf}
\usetikzlibrary{positioning,shapes,matrix,chains,scopes,fit,scopes,calc,shadings,intersections}
\usepackage[nomessages]{fp}

\newcommand{\railname}[1]{\it #1}
\newcommand{\railtermname}[1]{\bf #1}
\newlength\railcolsep
\setlength\railcolsep{5pt}
\newlength\railrowsep
\setlength\railrowsep{\baselineskip}
\newlength\railnodeheight
\setlength\railnodeheight{14pt}
\newlength\railcorners
\setlength\railcorners{\railcolsep}
\addtolength\railcorners{-1pt}

\tikzset{
  nonterminal/.style={
    draw,
    inner sep=2pt,
    % The shape:
    rectangle,
    % The size:
    minimum size=\railnodeheight,
    % The border:
    thick,
%    draw=red!50!black!50,         % 50% red and 50% black,
    % and that mixed with 50% white
    %shading=ball,
    %ball color=white,
     % The filling:
   top color=white,              % a shading that is white at the top...
    bottom color=red!50!black!20, % and something else at the bottom
    % Font
    font=\itshape
  },
  terminal/.style={
    draw,
    inner sep=2pt,
    % The shape:
    rounded rectangle,
    minimum size=\railnodeheight,
    % The rest
    thick,
%    draw=black!50,
    top color=white,bottom color=black!20,
    font=\ttfamily
  },
  every picture/.style={thick},%,rounded corners=0.001\railcolsep},
%  every picture/.style={thick,rounded corners={\railcorners}},%,rounded corners=0.001\railcolsep},
  every node/.style={inner sep=0,outer sep=0,draw=none}
%  skip loop/.style={to path={-- ++(0,#1) -| (\tikztotarget)}}
}

\makeatletter
\tikzset{
    getyshift/.style args={#1 and #2 and #3}{
      getdistc={#1}{#2},yshift=-\mylength-\railnodeheight+\pgflinewidth/#3
    },
getdistc/.code 2 args={
\pgfextra{
    \pgfpointdiff{\pgfpointanchor{#1}{north}}{\pgfpointanchor{#2}{south}}
    \xdef\mylength{\the\pgf@y}
         }
    }
}
\makeatother

\makeatletter
\newcommand{\Distance}[3]{% % from https://tex.stackexchange.com/q/56353/121799
\tikz@scan@one@point\pgfutil@firstofone($#1-#2$)\relax  
\pgfmathsetmacro{#3}{round(0.99626*veclen(\the\pgf@x,\the\pgf@y)/0.0283465)/1000}
}% Explanation: the calc library allows us, among other things, to add and
% subtract points, so ($#1-#2$) is simply the difference between the points
% #1 and #2. The combination \tikz@scan@one@point\pgfutil@firstofone extracts
% the coordinates of the new point and stores them in \pgf@x and \pgf@y. 
% They get fed in veclen, and \pgfmathsetmacro stores the result in #3. 
% EDIT: included fudge factor, see https://tex.stackexchange.com/a/22702/121799
\makeatother

\makeatletter
\newcommand\settoxdiff[3]{%
%\pgfpointdiff{\pgfpointanchor{#1}{center}}%
%             {\pgfpointanchor{#2}{center}}%
\pgfpointdiff{#2}{#3}%
%\pgf@xa=
\global\expandafter\edef #1{\pgf@x}%
%% \FP@eval\@temp@a{\pgfmath@tonumber{\pgf@xa}}%
%% \global\expandafter\edef\csname #3\endcsname{\pgf@xa}%
%\setlength{#3}{\pgf@x pt}%
}

\def\settodistance(#1,#2)#3{%
\pgfpointdiff{\pgfpointanchor{#1}{center}}%
             {\pgfpointanchor{#2}{center}}%
\pgf@xa=\pgf@x%
\pgf@ya=\pgf@y%
\global\expandafter\edef\csname xlen\endcsname{\pgf@xa}%
\global\expandafter\edef\csname ylen\endcsname{\pgf@ya}%
\FP@eval\@temp@a{\pgfmath@tonumber{\pgf@xa}}%
\FPeval\@temp@b{\pgfmath@tonumber{\pgf@ya}}%
\FPeval\@temp@sum{(\@temp@a*\@temp@a+\@temp@b*\@temp@b)}%
\FProot{\FPMathLen}{\@temp@sum}{2}%
\FPround\FPMathLen\FPMathLen5\relax%
\global\expandafter\edef\csname #3\endcsname{\pgf@x}%
}%
\makeatother




\newwrite\tempfile
\immediate\openout\tempfile=bnfnodes.dat


%\pgfpointdiff{\pgfpointanchor{TestBlock}{south east}}{\pgfpointanchor{TestBlock}{north east}}
%\pgfmathsetmacro\mytemp{\csname pgf@y\endcsname}

\makeatletter
\newcommand{\writenodesize}[1]{%
  \pgfpointdiff{\pgfpointanchor{#1}{west}}{\pgfpointanchor{#1}{east}}%
\FP@eval\@temp@x{\pgfmath@tonumber{\pgf@x}}%
  \pgfpointdiff{\pgfpointanchor{#1}{south}}{\pgfpointanchor{#1}{north}}%
\FP@eval\@temp@y{\pgfmath@tonumber{\pgf@y}}%
\immediate\write\tempfile{#1 \@temp@x,  \@temp@y}}
\makeatother


\begin{document}
\newlength{\tmplength}
\newlength{\tmplengthb}
\newlength{\tmplengthc}
\setlength{\tmplength}{12pt}
\setlength{\tmplengthb}{12pt}
\setlength{\tmplengthc}{12pt}




\begin{figure}
\centerline{
\begin{tikzpicture}
\node at (0pt,0pt)[anchor=west](name){\railname{simple\strut}};
\coordinate (node11) at (16pt,-21pt);
\coordinate (node11linetop) at (16pt,-27pt);
\coordinate (node11linebottom) at (16pt,-94pt);
\draw [rounded corners=\railcorners] (node11linetop) -- (node11linebottom);
\draw [rounded corners=\railcorners] (node11linetop) -- (node11) -- +(west:8pt);
\coordinate (node12) at (46pt,-21pt);
\coordinate (node12linetop) at (46pt,-27pt);
\coordinate (node12linebottom) at (46pt,-94pt);
\draw [rounded corners=\railcorners] (node12linetop) -- (node12linebottom);
\draw [rounded corners=\railcorners] (node12linetop) -- (node12) -- +(east:8pt);
\coordinate (coord1) at (31pt,-21pt);
\node (node1) at (24pt,-36pt)[anchor=west,nonterminal] {\railname{a\strut}};
\writenodesize{node1}
\node (node2) at (24pt,-58pt)[anchor=west,nonterminal] {\railname{b\strut}};
\writenodesize{node2}
\node (node4) at (24pt,-80pt)[anchor=west,nonterminal] {\railname{c\strut}};
\writenodesize{node4}
\node (node5) at (24pt,-102pt)[anchor=west,nonterminal] {\railname{d\strut}};
\writenodesize{node5}
\draw [rounded corners=\railcorners] (node11.east) -- (coord1);
\draw [rounded corners=\railcorners] (coord1) -- (node12.west);
\draw [rounded corners=\railcorners] (node1.west) -- (node1.west-|node11) -- (node11linetop);
\draw [rounded corners=\railcorners] (node2.west) -- (node2.west-|node11) -- (node11linetop);
\draw [rounded corners=\railcorners] (node4.west) -- (node4.west-|node11) -- (node11linetop);
\draw [rounded corners=\railcorners] (node5.west) -- (node5.west-|node11) -- (node11linetop);
\draw [rounded corners=\railcorners] (node1.east) -- (node1.east-|node12) -- (node12linetop);
\draw [rounded corners=\railcorners] (node2.east) -- (node2.east-|node12) -- (node12linetop);
\draw [rounded corners=\railcorners] (node4.east) -- (node4.east-|node12) -- (node12linetop);
\draw [rounded corners=\railcorners] (node5.east) -- (node5.east-|node12) -- (node12linetop);
\end{tikzpicture}
}
\caption{No Caption.}
\label{No Caption.}
\end{figure}

\begin{figure}
\centerline{
\begin{tikzpicture}
\node at (0pt,0pt)[anchor=west](name){\railname{simple\strut}};
\coordinate (node22) at (16pt,-21pt);
\coordinate (node22linetop) at (16pt,-27pt);
\coordinate (node22linebottom) at (16pt,-35pt);
\draw [rounded corners=\railcorners] (node22linetop) -- (node22linebottom);
\draw [rounded corners=\railcorners] (node22linetop) -- (node22) -- +(east:8pt);
\coordinate (node23) at (46pt,-21pt);
\coordinate (node23linetop) at (46pt,-27pt);
\coordinate (node23linebottom) at (46pt,-35pt);
\draw [rounded corners=\railcorners] (node23linetop) -- (node23linebottom);
\draw [rounded corners=\railcorners] (node23linetop) -- (node23) -- +(west:8pt);
\node (node17) at (24pt,-21pt)[anchor=west,nonterminal] {\railname{b\strut}};
\writenodesize{node17}
\node (node16) at (24pt,-43pt)[anchor=west,nonterminal] {\railname{a\strut}};
\writenodesize{node16}
\draw [rounded corners=\railcorners] (node22.east) -- (node17.west);
\draw [rounded corners=\railcorners] (node17.east) -- (node23.west);
\draw [rounded corners=\railcorners] (node16.west) -- (node16.west-|node22) -- (node22linetop);
\draw [rounded corners=\railcorners] (node16.east) -- (node16.east-|node23) -- (node23linetop);
\end{tikzpicture}
}
\caption{No Caption.}
\label{No Caption.}
\end{figure}

\begin{figure}
\centerline{
\begin{tikzpicture}
\node at (0pt,0pt)[anchor=west](name){\railname{optionchoice\strut}};
\coordinate (node36) at (38pt,-21pt);
\coordinate (node36linetop) at (38pt,-27pt);
\coordinate (node36linebottom) at (38pt,-50pt);
\draw [rounded corners=\railcorners] (node36linetop) -- (node36linebottom);
\draw [rounded corners=\railcorners] (node36linetop) -- (node36) -- +(west:8pt);
\coordinate (node37) at (68pt,-21pt);
\coordinate (node37linetop) at (68pt,-27pt);
\coordinate (node37linebottom) at (68pt,-50pt);
\draw [rounded corners=\railcorners] (node37linetop) -- (node37linebottom);
\draw [rounded corners=\railcorners] (node37linetop) -- (node37) -- +(east:8pt);
\node (node27) at (16pt,-21pt)[anchor=west,nonterminal] {\railname{b\strut}};
\writenodesize{node27}
\draw [rounded corners=\railcorners] (node27.east) -- (node36.west);
\coordinate (coord3) at (53pt,-21pt);
\node (node28) at (46pt,-36pt)[anchor=west,nonterminal] {\railname{a\strut}};
\writenodesize{node28}
\node (node29) at (46pt,-58pt)[anchor=west,nonterminal] {\railname{b\strut}};
\writenodesize{node29}
\draw [rounded corners=\railcorners] (node36.east) -- (coord3);
\draw [rounded corners=\railcorners] (coord3) -- (node37.west);
\node (node40) at (76pt,-21pt)[anchor=west,nonterminal] {\railname{c\strut}};
\writenodesize{node40}
\draw [rounded corners=\railcorners] (node37.east) -- (node40.west);
\draw [rounded corners=\railcorners] (node28.west) -- (node28.west-|node36) -- (node36linetop);
\draw [rounded corners=\railcorners] (node29.west) -- (node29.west-|node36) -- (node36linetop);
\draw [rounded corners=\railcorners] (node28.east) -- (node28.east-|node37) -- (node37linetop);
\draw [rounded corners=\railcorners] (node29.east) -- (node29.east-|node37) -- (node37linetop);
\end{tikzpicture}
}
\caption{No Caption.}
\label{No Caption.}
\end{figure}

\begin{figure}
\centerline{
\begin{tikzpicture}
\node at (0pt,0pt)[anchor=west](name){\railname{case\_statement\strut}};
\coordinate (node47) at (16pt,-21pt);
\coordinate (node47linetop) at (16pt,-27pt);
\coordinate (node47linebottom) at (16pt,-28pt);
\draw [rounded corners=\railcorners] (node47linetop) -- (node47linebottom);
\draw [rounded corners=\railcorners] (node47linetop) -- (node47) -- +(west:8pt);
\coordinate (node48) at (78.14pt,-21pt);
\coordinate (node48linetop) at (78.14pt,-27pt);
\coordinate (node48linebottom) at (78.14pt,-28pt);
\draw [rounded corners=\railcorners] (node48linetop) -- (node48linebottom);
\draw [rounded corners=\railcorners] (node48linetop) -- (node48) -- +(east:8pt);
\coordinate (node60) at (197.185pt,-21pt);
\coordinate (node60linetop) at (197.185pt,-27pt);
\coordinate (node60linebottom) at (197.185pt,-121pt);
\draw [rounded corners=\railcorners] (node60linetop) -- (node60linebottom);
\draw [rounded corners=\railcorners] (node60linetop) -- (node60) -- +(east:8pt);
\coordinate (node633) at (248.553pt,-21pt);
\coordinate (node633linetop) at (248.553pt,-27pt);
\coordinate (node633linebottom) at (248.553pt,-101pt);
\draw [rounded corners=\railcorners] (node633linetop) -- (node633linebottom);
\draw [rounded corners=\railcorners] (node633linetop) -- (node633) -- +(east:8pt);
\coordinate (node636) at (256.553pt,-21pt);
\coordinate (node636linetop) at (256.553pt,-27pt);
\coordinate (node636linebottom) at (256.553pt,-79pt);
\draw [rounded corners=\railcorners] (node636linetop) -- (node636linebottom);
\draw [rounded corners=\railcorners] (node636linetop) -- (node636) -- +(west:8pt);
\coordinate (node642) at (345.303pt,-21pt);
\coordinate (node642linetop) at (345.303pt,-27pt);
\coordinate (node642linebottom) at (345.303pt,-79pt);
\draw [rounded corners=\railcorners] (node642linetop) -- (node642linebottom);
\draw [rounded corners=\railcorners] (node642linetop) -- (node642) -- +(east:8pt);
\coordinate (node644) at (353.303pt,-21pt);
\coordinate (node644linetop) at (353.303pt,-27pt);
\coordinate (node644linebottom) at (353.303pt,-101pt);
\draw [rounded corners=\railcorners] (node644linetop) -- (node644linebottom);
\draw [rounded corners=\railcorners] (node644linetop) -- (node644) -- +(west:8pt);
\coordinate (node61) at (490.18pt,-21pt);
\coordinate (node61linetop) at (490.18pt,-27pt);
\coordinate (node61linebottom) at (490.18pt,-121pt);
\draw [rounded corners=\railcorners] (node61linetop) -- (node61linebottom);
\draw [rounded corners=\railcorners] (node61linetop) -- (node61) -- +(west:8pt);
\coordinate (node68) at (568.506pt,-21pt);
\coordinate (node68linetop) at (568.506pt,-27pt);
\coordinate (node68linebottom) at (568.506pt,-28pt);
\draw [rounded corners=\railcorners] (node68linetop) -- (node68linebottom);
\draw [rounded corners=\railcorners] (node68linetop) -- (node68) -- +(west:8pt);
\coordinate (node69) at (606.646pt,-21pt);
\coordinate (node69linetop) at (606.646pt,-27pt);
\coordinate (node69linebottom) at (606.646pt,-28pt);
\draw [rounded corners=\railcorners] (node69linetop) -- (node69linebottom);
\draw [rounded corners=\railcorners] (node69linetop) -- (node69) -- +(east:8pt);
\coordinate (coord4) at (47.07pt,-21pt);
\node (node42) at (24pt,-36pt)[anchor=west,nonterminal] {\railname{label\strut}};
\writenodesize{node42}
\node (node43) at (54.14pt,-36pt)[anchor=west,terminal] {\railtermname{:\strut}};
\writenodesize{node43}
\draw [rounded corners=\railcorners] (node42.east) -- (node43.west);
\draw [rounded corners=\railcorners] (node47.east) -- (coord4);
\draw [rounded corners=\railcorners] (coord4) -- (node48.west);
\node (node50) at (86.14pt,-21pt)[anchor=west,terminal] {\railtermname{case\strut}};
\writenodesize{node50}
\draw [rounded corners=\railcorners] (node48.east) -- (node50.west);
\node (node51) at (121.978pt,-21pt)[anchor=west,nonterminal] {\railname{expression\strut}};
\writenodesize{node51}
\draw [rounded corners=\railcorners] (node50.east) -- (node51.west);
\node (node52) at (173.048pt,-21pt)[anchor=west,terminal] {\railtermname{is\strut}};
\writenodesize{node52}
\draw [rounded corners=\railcorners] (node51.east) -- (node52.west);
\draw [rounded corners=\railcorners] (node52.east) -- (node60.west);
\node (node620) at (205.185pt,-21pt)[anchor=west,terminal] {\railtermname{when\strut}};
\writenodesize{node620}
\node (node638) at (264.553pt,-21pt)[anchor=west,nonterminal] {\railname{simple\_expression\strut}};
\writenodesize{node638}
\node (node639) at (272.783pt,-43pt)[anchor=west,nonterminal] {\railname{discrete\_range\strut}};
\writenodesize{node639}
\node (node640) at (273.703pt,-65pt)[anchor=west,nonterminal] {\railname{simple\_name\strut}};
\writenodesize{node640}
\node (node641) at (282.014pt,-87pt)[anchor=west,terminal] {\railtermname{others\strut}};
\writenodesize{node641}
\draw [rounded corners=\railcorners] (node636.east) -- (node638.west);
\draw [rounded corners=\railcorners] (node638.east) -- (node642.west);
\node (node643) at (293.928pt,-109pt)[anchor=west,terminal] {\railtermname{|}};
\writenodesize{node643}
\draw [rounded corners=\railcorners] (node633.east) -- (node636.west);
\draw [rounded corners=\railcorners] (node642.east) -- (node644.west);
\draw [rounded corners=\railcorners] (node620.east) -- (node633.west);
\node (node622) at (361.303pt,-21pt)[anchor=west,terminal] {\railtermname{=>\strut}};
\writenodesize{node622}
\draw [rounded corners=\railcorners] (node644.east) -- (node622.west);
\node (node623) at (388.961pt,-21pt)[anchor=west,nonterminal] {\railname{sequence\_of\_statements\strut}};
\writenodesize{node623}
\draw [rounded corners=\railcorners] (node622.east) -- (node623.west);
\node (node54) at (335.683pt,-129pt)[anchor=west,terminal] {\railtermname{,\strut}};
\writenodesize{node54}
\draw [rounded corners=\railcorners] (node60.east) -- (node620.west);
\draw [rounded corners=\railcorners] (node623.east) -- (node61.west);
\node (node63) at (498.18pt,-21pt)[anchor=west,terminal] {\railtermname{end\strut}};
\writenodesize{node63}
\draw [rounded corners=\railcorners] (node61.east) -- (node63.west);
\node (node64) at (532.668pt,-21pt)[anchor=west,terminal] {\railtermname{case\strut}};
\writenodesize{node64}
\draw [rounded corners=\railcorners] (node63.east) -- (node64.west);
\draw [rounded corners=\railcorners] (node64.east) -- (node68.west);
\coordinate (coord6) at (587.576pt,-21pt);
\node (node65) at (576.506pt,-36pt)[anchor=west,nonterminal] {\railname{label\strut}};
\writenodesize{node65}
\draw [rounded corners=\railcorners] (node68.east) -- (coord6);
\draw [rounded corners=\railcorners] (coord6) -- (node69.west);
\node (node71) at (614.646pt,-21pt)[anchor=west,terminal] {\railtermname{;\strut}};
\writenodesize{node71}
\draw [rounded corners=\railcorners] (node69.east) -- (node71.west);
\draw [rounded corners=\railcorners] (node42.west) -- (node42.west-|node47) -- (node47linetop);
\draw [rounded corners=\railcorners] (node42.west) -- (node42.west-|node47) -- (node47linetop);
\draw [rounded corners=\railcorners] (node639.west) -- (node639.west-|node636) -- (node636linetop);
\draw [rounded corners=\railcorners] (node640.west) -- (node640.west-|node636) -- (node636linetop);
\draw [rounded corners=\railcorners] (node641.west) -- (node641.west-|node636) -- (node636linetop);
\draw [rounded corners=\railcorners] (node643.west) -- (node643.west-|node633) -- (node633linetop);
\draw [rounded corners=\railcorners] (node54.west) -- (node54.west-|node60) -- (node60linetop);
\draw [rounded corners=\railcorners] (node65.west) -- (node65.west-|node68) -- (node68linetop);
\draw [rounded corners=\railcorners] (node43.east) -- (node43.east-|node48) -- (node48linetop);
\draw [rounded corners=\railcorners] (node43.east) -- (node43.east-|node48) -- (node48linetop);
\draw [rounded corners=\railcorners] (node639.east) -- (node639.east-|node642) -- (node642linetop);
\draw [rounded corners=\railcorners] (node640.east) -- (node640.east-|node642) -- (node642linetop);
\draw [rounded corners=\railcorners] (node641.east) -- (node641.east-|node642) -- (node642linetop);
\draw [rounded corners=\railcorners] (node643.east) -- (node643.east-|node644) -- (node644linetop);
\draw [rounded corners=\railcorners] (node54.east) -- (node54.east-|node61) -- (node61linetop);
\draw [rounded corners=\railcorners] (node65.east) -- (node65.east-|node69) -- (node69linetop);
\end{tikzpicture}
}
\caption{No Caption.}
\label{No Caption.}
\end{figure}

\begin{figure}
\centerline{
\begin{tikzpicture}
\node at (0pt,0pt)[anchor=west](name){\railname{case\_statement\_alternative\strut}};
\coordinate (node646) at (59.3677pt,-21pt);
\coordinate (node646linetop) at (59.3677pt,-27pt);
\coordinate (node646linebottom) at (59.3677pt,-101pt);
\draw [rounded corners=\railcorners] (node646linetop) -- (node646linebottom);
\draw [rounded corners=\railcorners] (node646linetop) -- (node646) -- +(east:8pt);
\coordinate (node649) at (67.3677pt,-21pt);
\coordinate (node649linetop) at (67.3677pt,-27pt);
\coordinate (node649linebottom) at (67.3677pt,-79pt);
\draw [rounded corners=\railcorners] (node649linetop) -- (node649linebottom);
\draw [rounded corners=\railcorners] (node649linetop) -- (node649) -- +(west:8pt);
\coordinate (node655) at (156.118pt,-21pt);
\coordinate (node655linetop) at (156.118pt,-27pt);
\coordinate (node655linebottom) at (156.118pt,-79pt);
\draw [rounded corners=\railcorners] (node655linetop) -- (node655linebottom);
\draw [rounded corners=\railcorners] (node655linetop) -- (node655) -- +(east:8pt);
\coordinate (node657) at (164.118pt,-21pt);
\coordinate (node657linetop) at (164.118pt,-27pt);
\coordinate (node657linebottom) at (164.118pt,-101pt);
\draw [rounded corners=\railcorners] (node657linetop) -- (node657linebottom);
\draw [rounded corners=\railcorners] (node657linetop) -- (node657) -- +(west:8pt);
\node (node73) at (16pt,-21pt)[anchor=west,terminal] {\railtermname{when\strut}};
\writenodesize{node73}
\node (node651) at (75.3677pt,-21pt)[anchor=west,nonterminal] {\railname{simple\_expression\strut}};
\writenodesize{node651}
\node (node652) at (83.5977pt,-43pt)[anchor=west,nonterminal] {\railname{discrete\_range\strut}};
\writenodesize{node652}
\node (node653) at (84.5176pt,-65pt)[anchor=west,nonterminal] {\railname{simple\_name\strut}};
\writenodesize{node653}
\node (node654) at (92.8288pt,-87pt)[anchor=west,terminal] {\railtermname{others\strut}};
\writenodesize{node654}
\draw [rounded corners=\railcorners] (node649.east) -- (node651.west);
\draw [rounded corners=\railcorners] (node651.east) -- (node655.west);
\node (node656) at (104.743pt,-109pt)[anchor=west,terminal] {\railtermname{|}};
\writenodesize{node656}
\draw [rounded corners=\railcorners] (node646.east) -- (node649.west);
\draw [rounded corners=\railcorners] (node655.east) -- (node657.west);
\draw [rounded corners=\railcorners] (node73.east) -- (node646.west);
\node (node76) at (172.118pt,-21pt)[anchor=west,terminal] {\railtermname{=>\strut}};
\writenodesize{node76}
\draw [rounded corners=\railcorners] (node657.east) -- (node76.west);
\node (node77) at (199.775pt,-21pt)[anchor=west,nonterminal] {\railname{sequence\_of\_statements\strut}};
\writenodesize{node77}
\draw [rounded corners=\railcorners] (node76.east) -- (node77.west);
\draw [rounded corners=\railcorners] (node652.west) -- (node652.west-|node649) -- (node649linetop);
\draw [rounded corners=\railcorners] (node653.west) -- (node653.west-|node649) -- (node649linetop);
\draw [rounded corners=\railcorners] (node654.west) -- (node654.west-|node649) -- (node649linetop);
\draw [rounded corners=\railcorners] (node656.west) -- (node656.west-|node646) -- (node646linetop);
\draw [rounded corners=\railcorners] (node652.east) -- (node652.east-|node655) -- (node655linetop);
\draw [rounded corners=\railcorners] (node653.east) -- (node653.east-|node655) -- (node655linetop);
\draw [rounded corners=\railcorners] (node654.east) -- (node654.east-|node655) -- (node655linetop);
\draw [rounded corners=\railcorners] (node656.east) -- (node656.east-|node657) -- (node657linetop);
\end{tikzpicture}
}
\caption{No Caption.}
\label{No Caption.}
\end{figure}

\begin{figure}
\centerline{
\begin{tikzpicture}
\node at (0pt,0pt)[anchor=west](name){\railname{character\_literal\strut}};
\node (node79) at (16pt,-21pt)[anchor=west,terminal] {\railtermname{'\strut}};
\writenodesize{node79}
\node (node80) at (40pt,-21pt)[anchor=west,nonterminal] {\railname{graphic\_character\strut}};
\writenodesize{node80}
\draw [rounded corners=\railcorners] (node79.east) -- (node80.west);
\node (node82) at (117.09pt,-21pt)[anchor=west,terminal] {\railtermname{'\strut}};
\writenodesize{node82}
\draw [rounded corners=\railcorners] (node80.east) -- (node82.west);
\end{tikzpicture}
}
\caption{No Caption.}
\label{No Caption.}
\end{figure}

\begin{figure}
\centerline{
\begin{tikzpicture}
\node at (0pt,0pt)[anchor=west](name){\railname{choice\strut}};
\coordinate (node89) at (16pt,-21pt);
\coordinate (node89linetop) at (16pt,-27pt);
\coordinate (node89linebottom) at (16pt,-79pt);
\draw [rounded corners=\railcorners] (node89linetop) -- (node89linebottom);
\draw [rounded corners=\railcorners] (node89linetop) -- (node89) -- +(west:8pt);
\coordinate (node90) at (104.75pt,-21pt);
\coordinate (node90linetop) at (104.75pt,-27pt);
\coordinate (node90linebottom) at (104.75pt,-79pt);
\draw [rounded corners=\railcorners] (node90linetop) -- (node90linebottom);
\draw [rounded corners=\railcorners] (node90linetop) -- (node90) -- +(east:8pt);
\node (node84) at (24pt,-21pt)[anchor=west,nonterminal] {\railname{simple\_expression\strut}};
\writenodesize{node84}
\node (node85) at (32.23pt,-43pt)[anchor=west,nonterminal] {\railname{discrete\_range\strut}};
\writenodesize{node85}
\node (node87) at (33.15pt,-65pt)[anchor=west,nonterminal] {\railname{simple\_name\strut}};
\writenodesize{node87}
\node (node88) at (41.4611pt,-87pt)[anchor=west,terminal] {\railtermname{others\strut}};
\writenodesize{node88}
\draw [rounded corners=\railcorners] (node89.east) -- (node84.west);
\draw [rounded corners=\railcorners] (node84.east) -- (node90.west);
\draw [rounded corners=\railcorners] (node85.west) -- (node85.west-|node89) -- (node89linetop);
\draw [rounded corners=\railcorners] (node87.west) -- (node87.west-|node89) -- (node89linetop);
\draw [rounded corners=\railcorners] (node88.west) -- (node88.west-|node89) -- (node89linetop);
\draw [rounded corners=\railcorners] (node85.east) -- (node85.east-|node90) -- (node90linetop);
\draw [rounded corners=\railcorners] (node87.east) -- (node87.east-|node90) -- (node90linetop);
\draw [rounded corners=\railcorners] (node88.east) -- (node88.east-|node90) -- (node90linetop);
\end{tikzpicture}
}
\caption{No Caption.}
\label{No Caption.}
\end{figure}

\begin{figure}
\centerline{
\begin{tikzpicture}
\node at (0pt,0pt)[anchor=west](name){\railname{choices\strut}};
\coordinate (node100) at (16pt,-21pt);
\coordinate (node100linetop) at (16pt,-27pt);
\coordinate (node100linebottom) at (16pt,-101pt);
\draw [rounded corners=\railcorners] (node100linetop) -- (node100linebottom);
\draw [rounded corners=\railcorners] (node100linetop) -- (node100) -- +(east:8pt);
\coordinate (node625) at (24pt,-21pt);
\coordinate (node625linetop) at (24pt,-27pt);
\coordinate (node625linebottom) at (24pt,-79pt);
\draw [rounded corners=\railcorners] (node625linetop) -- (node625linebottom);
\draw [rounded corners=\railcorners] (node625linetop) -- (node625) -- +(west:8pt);
\coordinate (node631) at (112.75pt,-21pt);
\coordinate (node631linetop) at (112.75pt,-27pt);
\coordinate (node631linebottom) at (112.75pt,-79pt);
\draw [rounded corners=\railcorners] (node631linetop) -- (node631linebottom);
\draw [rounded corners=\railcorners] (node631linetop) -- (node631) -- +(east:8pt);
\coordinate (node101) at (120.75pt,-21pt);
\coordinate (node101linetop) at (120.75pt,-27pt);
\coordinate (node101linebottom) at (120.75pt,-101pt);
\draw [rounded corners=\railcorners] (node101linetop) -- (node101linebottom);
\draw [rounded corners=\railcorners] (node101linetop) -- (node101) -- +(west:8pt);
\node (node627) at (32pt,-21pt)[anchor=west,nonterminal] {\railname{simple\_expression\strut}};
\writenodesize{node627}
\node (node628) at (40.23pt,-43pt)[anchor=west,nonterminal] {\railname{discrete\_range\strut}};
\writenodesize{node628}
\node (node629) at (41.15pt,-65pt)[anchor=west,nonterminal] {\railname{simple\_name\strut}};
\writenodesize{node629}
\node (node630) at (49.4611pt,-87pt)[anchor=west,terminal] {\railtermname{others\strut}};
\writenodesize{node630}
\draw [rounded corners=\railcorners] (node625.east) -- (node627.west);
\draw [rounded corners=\railcorners] (node627.east) -- (node631.west);
\node (node94) at (61.3749pt,-109pt)[anchor=west,terminal] {\railtermname{|}};
\writenodesize{node94}
\draw [rounded corners=\railcorners] (node100.east) -- (node625.west);
\draw [rounded corners=\railcorners] (node631.east) -- (node101.west);
\draw [rounded corners=\railcorners] (node628.west) -- (node628.west-|node625) -- (node625linetop);
\draw [rounded corners=\railcorners] (node629.west) -- (node629.west-|node625) -- (node625linetop);
\draw [rounded corners=\railcorners] (node630.west) -- (node630.west-|node625) -- (node625linetop);
\draw [rounded corners=\railcorners] (node94.west) -- (node94.west-|node100) -- (node100linetop);
\draw [rounded corners=\railcorners] (node628.east) -- (node628.east-|node631) -- (node631linetop);
\draw [rounded corners=\railcorners] (node629.east) -- (node629.east-|node631) -- (node631linetop);
\draw [rounded corners=\railcorners] (node630.east) -- (node630.east-|node631) -- (node631linetop);
\draw [rounded corners=\railcorners] (node94.east) -- (node94.east-|node101) -- (node101linetop);
\end{tikzpicture}
}
\caption{No Caption.}
\label{No Caption.}
\end{figure}

\begin{figure}
\centerline{
\begin{tikzpicture}
\node at (0pt,0pt)[anchor=west](name){\railname{toollist\strut}};
\coordinate (node112) at (16pt,-21pt);
\coordinate (node112linetop) at (16pt,-27pt);
\coordinate (node112linebottom) at (16pt,-123pt);
\draw [rounded corners=\railcorners] (node112linetop) -- (node112linebottom);
\draw [rounded corners=\railcorners] (node112linetop) -- (node112) -- +(east:8pt);
\coordinate (node659) at (24pt,-21pt);
\coordinate (node659linetop) at (24pt,-27pt);
\coordinate (node659linebottom) at (24pt,-57pt);
\draw [rounded corners=\railcorners] (node659linetop) -- (node659linebottom);
\draw [rounded corners=\railcorners] (node659linetop) -- (node659) -- +(west:8pt);
\coordinate (node674) at (85.0476pt,-87pt);
\coordinate (node674linetop) at (85.0476pt,-93pt);
\coordinate (node674linebottom) at (85.0476pt,-94pt);
\draw [rounded corners=\railcorners] (node674linetop) -- (node674linebottom);
\draw [rounded corners=\railcorners] (node674linetop) -- (node674) -- +(east:8pt);
\coordinate (node675) at (93.0476pt,-87pt);
\coordinate (node675linetop) at (93.0476pt,-93pt);
\coordinate (node675linebottom) at (93.0476pt,-101pt);
\draw [rounded corners=\railcorners] (node675linetop) -- (node675linebottom);
\draw [rounded corners=\railcorners] (node675linetop) -- (node675) -- +(west:8pt);
\coordinate (node679) at (179.545pt,-87pt);
\coordinate (node679linetop) at (179.545pt,-93pt);
\coordinate (node679linebottom) at (179.545pt,-101pt);
\draw [rounded corners=\railcorners] (node679linetop) -- (node679linebottom);
\draw [rounded corners=\railcorners] (node679linetop) -- (node679) -- +(east:8pt);
\coordinate (node681) at (233.163pt,-65pt);
\coordinate (node681linetop) at (233.163pt,-71pt);
\coordinate (node681linebottom) at (233.163pt,-79pt);
\draw [rounded corners=\railcorners] (node681linetop) -- (node681linebottom);
\draw [rounded corners=\railcorners] (node681linetop) -- (node681) -- +(east:8pt);
\coordinate (node664) at (275.421pt,-21pt);
\coordinate (node664linetop) at (275.421pt,-27pt);
\coordinate (node664linebottom) at (275.421pt,-57pt);
\draw [rounded corners=\railcorners] (node664linetop) -- (node664linebottom);
\draw [rounded corners=\railcorners] (node664linetop) -- (node664) -- +(east:8pt);
\coordinate (node113) at (283.421pt,-21pt);
\coordinate (node113linetop) at (283.421pt,-27pt);
\coordinate (node113linebottom) at (283.421pt,-123pt);
\draw [rounded corners=\railcorners] (node113linetop) -- (node113linebottom);
\draw [rounded corners=\railcorners] (node113linetop) -- (node113) -- +(west:8pt);
\node (node661) at (125.541pt,-21pt)[anchor=west,terminal] {\railtermname{hammer\strut}};
\writenodesize{node661}
\node (node662) at (118.741pt,-43pt)[anchor=west,terminal] {\railtermname{screwdriver\strut}};
\writenodesize{node662}
\node (node668) at (112.098pt,-65pt)[anchor=west,terminal] {\railtermname{hand\strut}};
\writenodesize{node668}
\coordinate (node672) at (54.5238pt,-87pt);
\node (node673) at (32pt,-102pt)[anchor=west,terminal] {\railtermname{cordless\strut}};
\writenodesize{node673}
\draw [rounded corners=\railcorners] (node672.east) -- (node674.west);
\draw [rounded corners=\railcorners] (node674.east) -- (node675.west);
\node (node677) at (101.048pt,-87pt)[anchor=west,terminal] {\railtermname{reciprocating\strut}};
\writenodesize{node677}
\node (node678) at (114.288pt,-109pt)[anchor=west,terminal] {\railtermname{circular\strut}};
\writenodesize{node678}
\draw [rounded corners=\railcorners] (node675.east) -- (node677.west);
\draw [rounded corners=\railcorners] (node677.east) -- (node679.west);
\node (node680) at (187.545pt,-87pt)[anchor=west,terminal] {\railtermname{power\strut}};
\writenodesize{node680}
\draw [rounded corners=\railcorners] (node679.east) -- (node680.west);
\draw [rounded corners=\railcorners] (node668.east) -- (node681.west);
\node (node682) at (241.163pt,-65pt)[anchor=west,terminal] {\railtermname{saw\strut}};
\writenodesize{node682}
\draw [rounded corners=\railcorners] (node681.east) -- (node682.west);
\draw [rounded corners=\railcorners] (node659.east) -- (node661.west);
\draw [rounded corners=\railcorners] (node661.east) -- (node664.west);
\node (node106) at (141.71pt,-131pt)[anchor=west,terminal] {\railtermname{,\strut}};
\writenodesize{node106}
\draw [rounded corners=\railcorners] (node112.east) -- (node659.west);
\draw [rounded corners=\railcorners] (node664.east) -- (node113.west);
\draw [rounded corners=\railcorners] (node662.west) -- (node662.west-|node659) -- (node659linetop);
\draw [rounded corners=\railcorners] (node668.west) -- (node668.west-|node659) -- (node659linetop);
\draw [rounded corners=\railcorners] (node668.west) -- (node668.west-|node659) -- (node659linetop);
\draw [rounded corners=\railcorners] (node672.west) -- (node672.west-|node659) -- (node659linetop);
\draw [rounded corners=\railcorners] (node672.west) -- (node672.west-|node659) -- (node659linetop);
\draw [rounded corners=\railcorners] (node673.west) -- (node673.west-|node659) -- (node659linetop);
\draw [rounded corners=\railcorners] (node673.west) -- (node673.west-|node659) -- (node659linetop);
\draw [rounded corners=\railcorners] (node678.west) -- (node678.west-|node675) -- (node675linetop);
\draw [rounded corners=\railcorners] (node672.west) -- (node672.west-|node659) -- (node659linetop);
\draw [rounded corners=\railcorners] (node672.west) -- (node672.west-|node659) -- (node659linetop);
\draw [rounded corners=\railcorners] (node673.west) -- (node673.west-|node659) -- (node659linetop);
\draw [rounded corners=\railcorners] (node673.west) -- (node673.west-|node659) -- (node659linetop);
\draw [rounded corners=\railcorners] (node678.west) -- (node678.west-|node675) -- (node675linetop);
\draw [rounded corners=\railcorners] (node106.west) -- (node106.west-|node112) -- (node112linetop);
\draw [rounded corners=\railcorners] (node662.east) -- (node662.east-|node664) -- (node664linetop);
\draw [rounded corners=\railcorners] (node682.east) -- (node682.east-|node664) -- (node664linetop);
\draw [rounded corners=\railcorners] (node682.east) -- (node682.east-|node664) -- (node664linetop);
\draw [rounded corners=\railcorners] (node680.east) -- (node680.east-|node681) -- (node681linetop);
\draw [rounded corners=\railcorners] (node680.east) -- (node680.east-|node681) -- (node681linetop);
\draw [rounded corners=\railcorners] (node673.east) -- (node673.east-|node674) -- (node674linetop);
\draw [rounded corners=\railcorners] (node678.east) -- (node678.east-|node679) -- (node679linetop);
\draw [rounded corners=\railcorners] (node106.east) -- (node106.east-|node113) -- (node113linetop);
\end{tikzpicture}
}
\caption{No Caption.}
\label{No Caption.}
\end{figure}

\begin{figure}
\centerline{
\begin{tikzpicture}
\node at (0pt,0pt)[anchor=west](name){\railname{tool\strut}};
\coordinate (node121) at (16pt,-21pt);
\coordinate (node121linetop) at (16pt,-27pt);
\coordinate (node121linebottom) at (16pt,-57pt);
\draw [rounded corners=\railcorners] (node121linetop) -- (node121linebottom);
\draw [rounded corners=\railcorners] (node121linetop) -- (node121) -- +(west:8pt);
\coordinate (node692) at (77.0476pt,-87pt);
\coordinate (node692linetop) at (77.0476pt,-93pt);
\coordinate (node692linebottom) at (77.0476pt,-94pt);
\draw [rounded corners=\railcorners] (node692linetop) -- (node692linebottom);
\draw [rounded corners=\railcorners] (node692linetop) -- (node692) -- +(east:8pt);
\coordinate (node693) at (85.0476pt,-87pt);
\coordinate (node693linetop) at (85.0476pt,-93pt);
\coordinate (node693linebottom) at (85.0476pt,-101pt);
\draw [rounded corners=\railcorners] (node693linetop) -- (node693linebottom);
\draw [rounded corners=\railcorners] (node693linetop) -- (node693) -- +(west:8pt);
\coordinate (node697) at (171.545pt,-87pt);
\coordinate (node697linetop) at (171.545pt,-93pt);
\coordinate (node697linebottom) at (171.545pt,-101pt);
\draw [rounded corners=\railcorners] (node697linetop) -- (node697linebottom);
\draw [rounded corners=\railcorners] (node697linetop) -- (node697) -- +(east:8pt);
\coordinate (node699) at (225.163pt,-65pt);
\coordinate (node699linetop) at (225.163pt,-71pt);
\coordinate (node699linebottom) at (225.163pt,-79pt);
\draw [rounded corners=\railcorners] (node699linetop) -- (node699linebottom);
\draw [rounded corners=\railcorners] (node699linetop) -- (node699) -- +(east:8pt);
\coordinate (node122) at (267.421pt,-21pt);
\coordinate (node122linetop) at (267.421pt,-27pt);
\coordinate (node122linebottom) at (267.421pt,-57pt);
\draw [rounded corners=\railcorners] (node122linetop) -- (node122linebottom);
\draw [rounded corners=\railcorners] (node122linetop) -- (node122) -- +(east:8pt);
\node (node117) at (117.541pt,-21pt)[anchor=west,terminal] {\railtermname{hammer\strut}};
\writenodesize{node117}
\node (node118) at (110.741pt,-43pt)[anchor=west,terminal] {\railtermname{screwdriver\strut}};
\writenodesize{node118}
\node (node686) at (104.098pt,-65pt)[anchor=west,terminal] {\railtermname{hand\strut}};
\writenodesize{node686}
\coordinate (node690) at (46.5238pt,-87pt);
\node (node691) at (24pt,-102pt)[anchor=west,terminal] {\railtermname{cordless\strut}};
\writenodesize{node691}
\draw [rounded corners=\railcorners] (node690.east) -- (node692.west);
\draw [rounded corners=\railcorners] (node692.east) -- (node693.west);
\node (node695) at (93.0476pt,-87pt)[anchor=west,terminal] {\railtermname{reciprocating\strut}};
\writenodesize{node695}
\node (node696) at (106.288pt,-109pt)[anchor=west,terminal] {\railtermname{circular\strut}};
\writenodesize{node696}
\draw [rounded corners=\railcorners] (node693.east) -- (node695.west);
\draw [rounded corners=\railcorners] (node695.east) -- (node697.west);
\node (node698) at (179.545pt,-87pt)[anchor=west,terminal] {\railtermname{power\strut}};
\writenodesize{node698}
\draw [rounded corners=\railcorners] (node697.east) -- (node698.west);
\draw [rounded corners=\railcorners] (node686.east) -- (node699.west);
\node (node700) at (233.163pt,-65pt)[anchor=west,terminal] {\railtermname{saw\strut}};
\writenodesize{node700}
\draw [rounded corners=\railcorners] (node699.east) -- (node700.west);
\draw [rounded corners=\railcorners] (node121.east) -- (node117.west);
\draw [rounded corners=\railcorners] (node117.east) -- (node122.west);
\draw [rounded corners=\railcorners] (node118.west) -- (node118.west-|node121) -- (node121linetop);
\draw [rounded corners=\railcorners] (node686.west) -- (node686.west-|node121) -- (node121linetop);
\draw [rounded corners=\railcorners] (node686.west) -- (node686.west-|node121) -- (node121linetop);
\draw [rounded corners=\railcorners] (node690.west) -- (node690.west-|node121) -- (node121linetop);
\draw [rounded corners=\railcorners] (node690.west) -- (node690.west-|node121) -- (node121linetop);
\draw [rounded corners=\railcorners] (node691.west) -- (node691.west-|node121) -- (node121linetop);
\draw [rounded corners=\railcorners] (node691.west) -- (node691.west-|node121) -- (node121linetop);
\draw [rounded corners=\railcorners] (node696.west) -- (node696.west-|node693) -- (node693linetop);
\draw [rounded corners=\railcorners] (node690.west) -- (node690.west-|node121) -- (node121linetop);
\draw [rounded corners=\railcorners] (node690.west) -- (node690.west-|node121) -- (node121linetop);
\draw [rounded corners=\railcorners] (node691.west) -- (node691.west-|node121) -- (node121linetop);
\draw [rounded corners=\railcorners] (node691.west) -- (node691.west-|node121) -- (node121linetop);
\draw [rounded corners=\railcorners] (node696.west) -- (node696.west-|node693) -- (node693linetop);
\draw [rounded corners=\railcorners] (node118.east) -- (node118.east-|node122) -- (node122linetop);
\draw [rounded corners=\railcorners] (node700.east) -- (node700.east-|node122) -- (node122linetop);
\draw [rounded corners=\railcorners] (node700.east) -- (node700.east-|node122) -- (node122linetop);
\draw [rounded corners=\railcorners] (node698.east) -- (node698.east-|node699) -- (node699linetop);
\draw [rounded corners=\railcorners] (node698.east) -- (node698.east-|node699) -- (node699linetop);
\draw [rounded corners=\railcorners] (node691.east) -- (node691.east-|node692) -- (node692linetop);
\draw [rounded corners=\railcorners] (node696.east) -- (node696.east-|node697) -- (node697linetop);
\end{tikzpicture}
}
\caption{No Caption.}
\label{No Caption.}
\end{figure}

\begin{figure}
\centerline{
\begin{tikzpicture}
\node at (0pt,0pt)[anchor=west](name){\railname{saw\strut}};
\coordinate (node140) at (16pt,-21pt);
\coordinate (node140linetop) at (16pt,-27pt);
\coordinate (node140linebottom) at (16pt,-35pt);
\draw [rounded corners=\railcorners] (node140linetop) -- (node140linebottom);
\draw [rounded corners=\railcorners] (node140linetop) -- (node140) -- +(west:8pt);
\coordinate (node130) at (77.0476pt,-43pt);
\coordinate (node130linetop) at (77.0476pt,-49pt);
\coordinate (node130linebottom) at (77.0476pt,-50pt);
\draw [rounded corners=\railcorners] (node130linetop) -- (node130linebottom);
\draw [rounded corners=\railcorners] (node130linetop) -- (node130) -- +(east:8pt);
\coordinate (node135) at (85.0476pt,-43pt);
\coordinate (node135linetop) at (85.0476pt,-49pt);
\coordinate (node135linebottom) at (85.0476pt,-57pt);
\draw [rounded corners=\railcorners] (node135linetop) -- (node135linebottom);
\draw [rounded corners=\railcorners] (node135linetop) -- (node135) -- +(west:8pt);
\coordinate (node136) at (171.545pt,-43pt);
\coordinate (node136linetop) at (171.545pt,-49pt);
\coordinate (node136linebottom) at (171.545pt,-57pt);
\draw [rounded corners=\railcorners] (node136linetop) -- (node136linebottom);
\draw [rounded corners=\railcorners] (node136linetop) -- (node136) -- +(east:8pt);
\coordinate (node141) at (225.163pt,-21pt);
\coordinate (node141linetop) at (225.163pt,-27pt);
\coordinate (node141linebottom) at (225.163pt,-35pt);
\draw [rounded corners=\railcorners] (node141linetop) -- (node141linebottom);
\draw [rounded corners=\railcorners] (node141linetop) -- (node141) -- +(east:8pt);
\node (node125) at (104.098pt,-21pt)[anchor=west,terminal] {\railtermname{hand\strut}};
\writenodesize{node125}
\coordinate (coord9) at (46.5238pt,-43pt);
\node (node126) at (24pt,-58pt)[anchor=west,terminal] {\railtermname{cordless\strut}};
\writenodesize{node126}
\draw [rounded corners=\railcorners] (coord9) -- (node130.west);
\draw [rounded corners=\railcorners] (node130.east) -- (node135.west);
\node (node132) at (93.0476pt,-43pt)[anchor=west,terminal] {\railtermname{reciprocating\strut}};
\writenodesize{node132}
\node (node133) at (106.288pt,-65pt)[anchor=west,terminal] {\railtermname{circular\strut}};
\writenodesize{node133}
\draw [rounded corners=\railcorners] (node135.east) -- (node132.west);
\draw [rounded corners=\railcorners] (node132.east) -- (node136.west);
\node (node138) at (179.545pt,-43pt)[anchor=west,terminal] {\railtermname{power\strut}};
\writenodesize{node138}
\draw [rounded corners=\railcorners] (node136.east) -- (node138.west);
\draw [rounded corners=\railcorners] (node140.east) -- (node125.west);
\draw [rounded corners=\railcorners] (node125.east) -- (node141.west);
\node (node143) at (233.163pt,-21pt)[anchor=west,terminal] {\railtermname{saw\strut}};
\writenodesize{node143}
\draw [rounded corners=\railcorners] (node141.east) -- (node143.west);
\draw [rounded corners=\railcorners] (coord9) -- (coord9-|node140) -- (node140linetop);
\draw [rounded corners=\railcorners] (coord9) -- (coord9-|node140) -- (node140linetop);
\draw [rounded corners=\railcorners] (node126.west) -- (node126.west-|node140) -- (node140linetop);
\draw [rounded corners=\railcorners] (node126.west) -- (node126.west-|node140) -- (node140linetop);
\draw [rounded corners=\railcorners] (node133.west) -- (node133.west-|node135) -- (node135linetop);
\draw [rounded corners=\railcorners] (node138.east) -- (node138.east-|node141) -- (node141linetop);
\draw [rounded corners=\railcorners] (node138.east) -- (node138.east-|node141) -- (node141linetop);
\draw [rounded corners=\railcorners] (node126.east) -- (node126.east-|node130) -- (node130linetop);
\draw [rounded corners=\railcorners] (node133.east) -- (node133.east-|node136) -- (node136linetop);
\end{tikzpicture}
}
\caption{No Caption.}
\label{No Caption.}
\end{figure}

\begin{figure}
\centerline{
\begin{tikzpicture}
\node at (0pt,0pt)[anchor=west](name){\railname{nolist\strut}};
\coordinate (node153) at (60pt,-21pt);
\coordinate (node153linetop) at (60pt,-27pt);
\coordinate (node153linebottom) at (60pt,-28pt);
\draw [rounded corners=\railcorners] (node153linetop) -- (node153linebottom);
\draw [rounded corners=\railcorners] (node153linetop) -- (node153) -- +(west:8pt);
\coordinate (node154) at (114pt,-21pt);
\coordinate (node154linetop) at (114pt,-27pt);
\coordinate (node154linebottom) at (114pt,-28pt);
\draw [rounded corners=\railcorners] (node154linetop) -- (node154linebottom);
\draw [rounded corners=\railcorners] (node154linetop) -- (node154) -- +(east:8pt);
\node (node145) at (16pt,-21pt)[anchor=west,nonterminal] {\railname{a\strut}};
\writenodesize{node145}
\node (node146) at (38pt,-21pt)[anchor=west,nonterminal] {\railname{b\strut}};
\writenodesize{node146}
\draw [rounded corners=\railcorners] (node145.east) -- (node146.west);
\draw [rounded corners=\railcorners] (node146.east) -- (node153.west);
\coordinate (coord10) at (87pt,-21pt);
\node (node148) at (68pt,-36pt)[anchor=west,nonterminal] {\railname{c\strut}};
\writenodesize{node148}
\node (node149) at (90pt,-36pt)[anchor=west,terminal] {\railtermname{d\strut}};
\writenodesize{node149}
\draw [rounded corners=\railcorners] (node148.east) -- (node149.west);
\draw [rounded corners=\railcorners] (node153.east) -- (coord10);
\draw [rounded corners=\railcorners] (coord10) -- (node154.west);
\node (node156) at (122pt,-21pt)[anchor=west,nonterminal] {\railname{e\strut}};
\writenodesize{node156}
\draw [rounded corners=\railcorners] (node154.east) -- (node156.west);
\node (node157) at (144pt,-21pt)[anchor=west,nonterminal] {\railname{f\strut}};
\writenodesize{node157}
\draw [rounded corners=\railcorners] (node156.east) -- (node157.west);
\draw [rounded corners=\railcorners] (node148.west) -- (node148.west-|node153) -- (node153linetop);
\draw [rounded corners=\railcorners] (node148.west) -- (node148.west-|node153) -- (node153linetop);
\draw [rounded corners=\railcorners] (node149.east) -- (node149.east-|node154) -- (node154linetop);
\draw [rounded corners=\railcorners] (node149.east) -- (node149.east-|node154) -- (node154linetop);
\end{tikzpicture}
}
\caption{No Caption.}
\label{No Caption.}
\end{figure}

\begin{figure}
\centerline{
\begin{tikzpicture}
\node at (0pt,0pt)[anchor=west](name){\railname{list\strut}};
\coordinate (node170) at (16pt,-21pt);
\coordinate (node170linetop) at (16pt,-27pt);
\coordinate (node170linebottom) at (16pt,-57pt);
\draw [rounded corners=\railcorners] (node170linetop) -- (node170linebottom);
\draw [rounded corners=\railcorners] (node170linetop) -- (node170) -- +(west:8pt);
\coordinate (node171) at (117.51pt,-21pt);
\coordinate (node171linetop) at (117.51pt,-27pt);
\coordinate (node171linebottom) at (117.51pt,-57pt);
\draw [rounded corners=\railcorners] (node171linetop) -- (node171linebottom);
\draw [rounded corners=\railcorners] (node171linetop) -- (node171) -- +(east:8pt);
\node (node159) at (24pt,-21pt)[anchor=west,nonterminal] {\railname{a\strut}};
\writenodesize{node159}
\node (node160) at (46pt,-21pt)[anchor=west,terminal] {\railtermname{b\strut}};
\writenodesize{node160}
\draw [rounded corners=\railcorners] (node159.east) -- (node160.west);
\node (node162) at (70pt,-21pt)[anchor=west,nonterminal] {\railname{c\strut}};
\writenodesize{node162}
\draw [rounded corners=\railcorners] (node160.east) -- (node162.west);
\node (node163) at (92pt,-21pt)[anchor=west,nonterminal] {\railname{this\strut}};
\writenodesize{node163}
\draw [rounded corners=\railcorners] (node162.east) -- (node163.west);
\node (node164) at (48.24pt,-43pt)[anchor=west,nonterminal] {\railname{that\_one\strut}};
\writenodesize{node164}
\node (node166) at (26.1662pt,-65pt)[anchor=west,terminal] {\railtermname{and\strut}};
\writenodesize{node166}
\node (node167) at (60.8238pt,-65pt)[anchor=west,nonterminal] {\railname{the\strut}};
\writenodesize{node167}
\draw [rounded corners=\railcorners] (node166.east) -- (node167.west);
\node (node169) at (84.2938pt,-65pt)[anchor=west,nonterminal] {\railname{other\strut}};
\writenodesize{node169}
\draw [rounded corners=\railcorners] (node167.east) -- (node169.west);
\draw [rounded corners=\railcorners] (node170.east) -- (node159.west);
\draw [rounded corners=\railcorners] (node163.east) -- (node171.west);
\draw [rounded corners=\railcorners] (node164.west) -- (node164.west-|node170) -- (node170linetop);
\draw [rounded corners=\railcorners] (node166.west) -- (node166.west-|node170) -- (node170linetop);
\draw [rounded corners=\railcorners] (node166.west) -- (node166.west-|node170) -- (node170linetop);
\draw [rounded corners=\railcorners] (node164.east) -- (node164.east-|node171) -- (node171linetop);
\draw [rounded corners=\railcorners] (node169.east) -- (node169.east-|node171) -- (node171linetop);
\draw [rounded corners=\railcorners] (node169.east) -- (node169.east-|node171) -- (node171linetop);
\end{tikzpicture}
}
\caption{No Caption.}
\label{No Caption.}
\end{figure}

\begin{figure}
\centerline{
\begin{tikzpicture}
\node at (0pt,0pt)[anchor=west](name){\railname{list\strut}};
\coordinate (node185) at (62pt,-21pt);
\coordinate (node185linetop) at (62pt,-27pt);
\coordinate (node185linebottom) at (62pt,-35pt);
\draw [rounded corners=\railcorners] (node185linetop) -- (node185linebottom);
\draw [rounded corners=\railcorners] (node185linetop) -- (node185) -- +(west:8pt);
\coordinate (node186) at (173.158pt,-21pt);
\coordinate (node186linetop) at (173.158pt,-27pt);
\coordinate (node186linebottom) at (173.158pt,-35pt);
\draw [rounded corners=\railcorners] (node186linetop) -- (node186linebottom);
\draw [rounded corners=\railcorners] (node186linetop) -- (node186) -- +(east:8pt);
\node (node174) at (16pt,-21pt)[anchor=west,nonterminal] {\railname{a\strut}};
\writenodesize{node174}
\node (node175) at (38pt,-21pt)[anchor=west,terminal] {\railtermname{b\strut}};
\writenodesize{node175}
\draw [rounded corners=\railcorners] (node174.east) -- (node175.west);
\draw [rounded corners=\railcorners] (node175.east) -- (node185.west);
\node (node177) at (97.8238pt,-21pt)[anchor=west,nonterminal] {\railname{c\strut}};
\writenodesize{node177}
\node (node178) at (119.824pt,-21pt)[anchor=west,nonterminal] {\railname{this\strut}};
\writenodesize{node178}
\draw [rounded corners=\railcorners] (node177.east) -- (node178.west);
\node (node180) at (70pt,-43pt)[anchor=west,nonterminal] {\railname{that\_one\strut}};
\writenodesize{node180}
\node (node181) at (115.03pt,-43pt)[anchor=west,terminal] {\railtermname{and\strut}};
\writenodesize{node181}
\draw [rounded corners=\railcorners] (node180.east) -- (node181.west);
\node (node183) at (149.688pt,-43pt)[anchor=west,nonterminal] {\railname{the\strut}};
\writenodesize{node183}
\draw [rounded corners=\railcorners] (node181.east) -- (node183.west);
\draw [rounded corners=\railcorners] (node185.east) -- (node177.west);
\draw [rounded corners=\railcorners] (node178.east) -- (node186.west);
\node (node188) at (181.158pt,-21pt)[anchor=west,nonterminal] {\railname{other\strut}};
\writenodesize{node188}
\draw [rounded corners=\railcorners] (node186.east) -- (node188.west);
\draw [rounded corners=\railcorners] (node180.west) -- (node180.west-|node185) -- (node185linetop);
\draw [rounded corners=\railcorners] (node180.west) -- (node180.west-|node185) -- (node185linetop);
\draw [rounded corners=\railcorners] (node183.east) -- (node183.east-|node186) -- (node186linetop);
\draw [rounded corners=\railcorners] (node183.east) -- (node183.east-|node186) -- (node186linetop);
\end{tikzpicture}
}
\caption{No Caption.}
\label{No Caption.}
\end{figure}

\begin{figure}
\centerline{
\begin{tikzpicture}
\node at (0pt,0pt)[anchor=west](name){\railname{listx\strut}};
\coordinate (node213) at (38pt,-21pt);
\coordinate (node213linetop) at (38pt,-27pt);
\coordinate (node213linebottom) at (38pt,-35pt);
\draw [rounded corners=\railcorners] (node213linetop) -- (node213linebottom);
\draw [rounded corners=\railcorners] (node213linetop) -- (node213) -- +(west:8pt);
\coordinate (node208) at (68pt,-43pt);
\coordinate (node208linetop) at (68pt,-49pt);
\coordinate (node208linebottom) at (68pt,-116pt);
\draw [rounded corners=\railcorners] (node208linetop) -- (node208linebottom);
\draw [rounded corners=\railcorners] (node208linetop) -- (node208) -- +(west:8pt);
\coordinate (node203) at (76pt,-65pt);
\coordinate (node203linetop) at (76pt,-71pt);
\coordinate (node203linebottom) at (76pt,-101pt);
\draw [rounded corners=\railcorners] (node203linetop) -- (node203linebottom);
\draw [rounded corners=\railcorners] (node203linetop) -- (node203) -- +(east:8pt);
\coordinate (node197) at (84pt,-65pt);
\coordinate (node197linetop) at (84pt,-71pt);
\coordinate (node197linebottom) at (84pt,-79pt);
\draw [rounded corners=\railcorners] (node197linetop) -- (node197linebottom);
\draw [rounded corners=\railcorners] (node197linetop) -- (node197) -- +(west:8pt);
\coordinate (node198) at (137.03pt,-65pt);
\coordinate (node198linetop) at (137.03pt,-71pt);
\coordinate (node198linebottom) at (137.03pt,-79pt);
\draw [rounded corners=\railcorners] (node198linetop) -- (node198linebottom);
\draw [rounded corners=\railcorners] (node198linetop) -- (node198) -- +(east:8pt);
\coordinate (node204) at (145.03pt,-65pt);
\coordinate (node204linetop) at (145.03pt,-71pt);
\coordinate (node204linebottom) at (145.03pt,-101pt);
\draw [rounded corners=\railcorners] (node204linetop) -- (node204linebottom);
\draw [rounded corners=\railcorners] (node204linetop) -- (node204) -- +(west:8pt);
\coordinate (node214) at (153.03pt,-21pt);
\coordinate (node214linetop) at (153.03pt,-27pt);
\coordinate (node214linebottom) at (153.03pt,-35pt);
\draw [rounded corners=\railcorners] (node214linetop) -- (node214linebottom);
\draw [rounded corners=\railcorners] (node214linetop) -- (node214) -- +(east:8pt);
\node (node190) at (16pt,-21pt)[anchor=west,nonterminal] {\railname{a\strut}};
\writenodesize{node190}
\draw [rounded corners=\railcorners] (node190.east) -- (node213.west);
\node (node191) at (87.515pt,-21pt)[anchor=west,terminal] {\railtermname{b\strut}};
\writenodesize{node191}
\node (node192) at (46pt,-43pt)[anchor=west,nonterminal] {\railname{c\strut}};
\writenodesize{node192}
\draw [rounded corners=\railcorners] (node192.east) -- (node208.west);
\node (node193) at (101.76pt,-43pt)[anchor=west,nonterminal] {\railname{this\strut}};
\writenodesize{node193}
\node (node194) at (92pt,-65pt)[anchor=west,nonterminal] {\railname{that\_one\strut}};
\writenodesize{node194}
\node (node195) at (97.1861pt,-87pt)[anchor=west,terminal] {\railtermname{and\strut}};
\writenodesize{node195}
\draw [rounded corners=\railcorners] (node197.east) -- (node194.west);
\draw [rounded corners=\railcorners] (node194.east) -- (node198.west);
\coordinate (coord11) at (110.515pt,-109pt);
\draw [rounded corners=\railcorners] (node203.east) -- (node197.west);
\draw [rounded corners=\railcorners] (node198.east) -- (node204.west);
\node (node207) at (102.78pt,-124pt)[anchor=west,nonterminal] {\railname{the\strut}};
\writenodesize{node207}
\draw [rounded corners=\railcorners] (node208.east) -- (node193.west);
\draw [rounded corners=\railcorners] (node213.east) -- (node191.west);
\draw [rounded corners=\railcorners] (node191.east) -- (node214.west);
\node (node217) at (161.03pt,-21pt)[anchor=west,nonterminal] {\railname{other\strut}};
\writenodesize{node217}
\draw [rounded corners=\railcorners] (node214.east) -- (node217.west);
\draw [rounded corners=\railcorners] (node192.west) -- (node192.west-|node213) -- (node213linetop);
\draw [rounded corners=\railcorners] (node192.west) -- (node192.west-|node213) -- (node213linetop);
\draw [rounded corners=\railcorners] (node203.west) -- (node203.west-|node208) -- (node208linetop);
\draw [rounded corners=\railcorners] (node195.west) -- (node195.west-|node197) -- (node197linetop);
\draw [rounded corners=\railcorners] (coord11) -- (coord11-|node203) -- (node203linetop);
\draw [rounded corners=\railcorners] (node207.west) -- (node207.west-|node208) -- (node208linetop);
\draw [rounded corners=\railcorners] (node193.east) -- (node193.east-|node214) -- (node214linetop);
\draw [rounded corners=\railcorners] (node193.east) -- (node193.east-|node214) -- (node214linetop);
\draw [rounded corners=\railcorners] (node204.east) -- (node204.east-|node214) -- (node214linetop);
\draw [rounded corners=\railcorners] (node195.east) -- (node195.east-|node198) -- (node198linetop);
\draw [rounded corners=\railcorners] (coord11) -- (coord11-|node204) -- (node204linetop);
\draw [rounded corners=\railcorners] (node207.east) -- (node207.east-|node214) -- (node214linetop);
\draw [rounded corners=\railcorners] (node204.east) -- (node204.east-|node214) -- (node214linetop);
\draw [rounded corners=\railcorners] (node195.east) -- (node195.east-|node198) -- (node198linetop);
\draw [rounded corners=\railcorners] (coord11) -- (coord11-|node204) -- (node204linetop);
\draw [rounded corners=\railcorners] (node207.east) -- (node207.east-|node214) -- (node214linetop);
\end{tikzpicture}
}
\caption{No Caption.}
\label{No Caption.}
\end{figure}

\begin{figure}
\centerline{
\begin{tikzpicture}
\node at (0pt,0pt)[anchor=west](name){\railname{listj\strut}};
\coordinate (node228) at (38pt,-21pt);
\coordinate (node228linetop) at (38pt,-27pt);
\coordinate (node228linebottom) at (38pt,-101pt);
\draw [rounded corners=\railcorners] (node228linetop) -- (node228linebottom);
\draw [rounded corners=\railcorners] (node228linetop) -- (node228) -- +(west:8pt);
\coordinate (node229) at (116.54pt,-21pt);
\coordinate (node229linetop) at (116.54pt,-27pt);
\coordinate (node229linebottom) at (116.54pt,-101pt);
\draw [rounded corners=\railcorners] (node229linetop) -- (node229linebottom);
\draw [rounded corners=\railcorners] (node229linetop) -- (node229) -- +(east:8pt);
\node (node219) at (16pt,-21pt)[anchor=west,nonterminal] {\railname{a\strut}};
\writenodesize{node219}
\draw [rounded corners=\railcorners] (node219.east) -- (node228.west);
\node (node220) at (69.2699pt,-21pt)[anchor=west,terminal] {\railtermname{b\strut}};
\writenodesize{node220}
\node (node221) at (70.2699pt,-43pt)[anchor=west,nonterminal] {\railname{c\strut}};
\writenodesize{node221}
\node (node223) at (46pt,-65pt)[anchor=west,nonterminal] {\railname{this\strut}};
\writenodesize{node223}
\node (node224) at (71.5099pt,-65pt)[anchor=west,nonterminal] {\railname{that\_one\strut}};
\writenodesize{node224}
\draw [rounded corners=\railcorners] (node223.east) -- (node224.west);
\node (node226) at (63.9411pt,-87pt)[anchor=west,terminal] {\railtermname{and\strut}};
\writenodesize{node226}
\node (node227) at (69.535pt,-109pt)[anchor=west,nonterminal] {\railname{the\strut}};
\writenodesize{node227}
\draw [rounded corners=\railcorners] (node228.east) -- (node220.west);
\draw [rounded corners=\railcorners] (node220.east) -- (node229.west);
\node (node232) at (124.54pt,-21pt)[anchor=west,nonterminal] {\railname{other\strut}};
\writenodesize{node232}
\draw [rounded corners=\railcorners] (node229.east) -- (node232.west);
\draw [rounded corners=\railcorners] (node221.west) -- (node221.west-|node228) -- (node228linetop);
\draw [rounded corners=\railcorners] (node223.west) -- (node223.west-|node228) -- (node228linetop);
\draw [rounded corners=\railcorners] (node223.west) -- (node223.west-|node228) -- (node228linetop);
\draw [rounded corners=\railcorners] (node226.west) -- (node226.west-|node228) -- (node228linetop);
\draw [rounded corners=\railcorners] (node227.west) -- (node227.west-|node228) -- (node228linetop);
\draw [rounded corners=\railcorners] (node221.east) -- (node221.east-|node229) -- (node229linetop);
\draw [rounded corners=\railcorners] (node224.east) -- (node224.east-|node229) -- (node229linetop);
\draw [rounded corners=\railcorners] (node224.east) -- (node224.east-|node229) -- (node229linetop);
\draw [rounded corners=\railcorners] (node226.east) -- (node226.east-|node229) -- (node229linetop);
\draw [rounded corners=\railcorners] (node227.east) -- (node227.east-|node229) -- (node229linetop);
\end{tikzpicture}
}
\caption{No Caption.}
\label{No Caption.}
\end{figure}

\begin{figure}
\centerline{
\begin{tikzpicture}
\node at (0pt,0pt)[anchor=west](name){\railname{listk\strut}};
\coordinate (node257) at (38pt,-21pt);
\coordinate (node257linetop) at (38pt,-27pt);
\coordinate (node257linebottom) at (38pt,-160pt);
\draw [rounded corners=\railcorners] (node257linetop) -- (node257linebottom);
\draw [rounded corners=\railcorners] (node257linetop) -- (node257) -- +(west:8pt);
\coordinate (node251) at (86.6977pt,-65pt);
\coordinate (node251linetop) at (86.6977pt,-71pt);
\coordinate (node251linebottom) at (86.6977pt,-123pt);
\draw [rounded corners=\railcorners] (node251linetop) -- (node251linebottom);
\draw [rounded corners=\railcorners] (node251linetop) -- (node251) -- +(east:8pt);
\coordinate (node245) at (94.6977pt,-65pt);
\coordinate (node245linetop) at (94.6977pt,-71pt);
\coordinate (node245linebottom) at (94.6977pt,-101pt);
\draw [rounded corners=\railcorners] (node245linetop) -- (node245linebottom);
\draw [rounded corners=\railcorners] (node245linetop) -- (node245) -- +(west:8pt);
\coordinate (node246) at (173.238pt,-65pt);
\coordinate (node246linetop) at (173.238pt,-71pt);
\coordinate (node246linebottom) at (173.238pt,-101pt);
\draw [rounded corners=\railcorners] (node246linetop) -- (node246linebottom);
\draw [rounded corners=\railcorners] (node246linetop) -- (node246) -- +(east:8pt);
\coordinate (node252) at (181.238pt,-65pt);
\coordinate (node252linetop) at (181.238pt,-71pt);
\coordinate (node252linebottom) at (181.238pt,-123pt);
\draw [rounded corners=\railcorners] (node252linetop) -- (node252linebottom);
\draw [rounded corners=\railcorners] (node252linetop) -- (node252) -- +(west:8pt);
\coordinate (node258) at (189.238pt,-21pt);
\coordinate (node258linetop) at (189.238pt,-27pt);
\coordinate (node258linebottom) at (189.238pt,-160pt);
\draw [rounded corners=\railcorners] (node258linetop) -- (node258linebottom);
\draw [rounded corners=\railcorners] (node258linetop) -- (node258) -- +(east:8pt);
\node (node234) at (16pt,-21pt)[anchor=west,nonterminal] {\railname{a\strut}};
\writenodesize{node234}
\draw [rounded corners=\railcorners] (node234.east) -- (node257.west);
\node (node235) at (105.619pt,-21pt)[anchor=west,terminal] {\railtermname{b\strut}};
\writenodesize{node235}
\node (node236) at (106.619pt,-43pt)[anchor=west,nonterminal] {\railname{c\strut}};
\writenodesize{node236}
\node (node238) at (46pt,-65pt)[anchor=west,terminal] {\railtermname{none\strut}};
\writenodesize{node238}
\draw [rounded corners=\railcorners] (node238.east) -- (node251.west);
\node (node239) at (102.698pt,-65pt)[anchor=west,nonterminal] {\railname{this\strut}};
\writenodesize{node239}
\node (node240) at (128.208pt,-65pt)[anchor=west,nonterminal] {\railname{that\_one\strut}};
\writenodesize{node240}
\draw [rounded corners=\railcorners] (node239.east) -- (node240.west);
\node (node242) at (113.393pt,-87pt)[anchor=west,nonterminal] {\railname{other\_one\strut}};
\writenodesize{node242}
\node (node244) at (117.619pt,-109pt)[anchor=west,terminal] {\railtermname{none\strut}};
\writenodesize{node244}
\draw [rounded corners=\railcorners] (node245.east) -- (node239.west);
\draw [rounded corners=\railcorners] (node240.east) -- (node246.west);
\coordinate (coord12) at (133.968pt,-131pt);
\draw [rounded corners=\railcorners] (node251.east) -- (node245.west);
\draw [rounded corners=\railcorners] (node246.east) -- (node252.west);
\node (node255) at (100.29pt,-146pt)[anchor=west,terminal] {\railtermname{and\strut}};
\writenodesize{node255}
\node (node256) at (105.884pt,-168pt)[anchor=west,nonterminal] {\railname{the\strut}};
\writenodesize{node256}
\draw [rounded corners=\railcorners] (node257.east) -- (node235.west);
\draw [rounded corners=\railcorners] (node235.east) -- (node258.west);
\node (node261) at (197.238pt,-21pt)[anchor=west,nonterminal] {\railname{other\strut}};
\writenodesize{node261}
\draw [rounded corners=\railcorners] (node258.east) -- (node261.west);
\draw [rounded corners=\railcorners] (node236.west) -- (node236.west-|node257) -- (node257linetop);
\draw [rounded corners=\railcorners] (node238.west) -- (node238.west-|node257) -- (node257linetop);
\draw [rounded corners=\railcorners] (node238.west) -- (node238.west-|node257) -- (node257linetop);
\draw [rounded corners=\railcorners] (node242.west) -- (node242.west-|node245) -- (node245linetop);
\draw [rounded corners=\railcorners] (node244.west) -- (node244.west-|node245) -- (node245linetop);
\draw [rounded corners=\railcorners] (coord12) -- (coord12-|node251) -- (node251linetop);
\draw [rounded corners=\railcorners] (node255.west) -- (node255.west-|node257) -- (node257linetop);
\draw [rounded corners=\railcorners] (node256.west) -- (node256.west-|node257) -- (node257linetop);
\draw [rounded corners=\railcorners] (node236.east) -- (node236.east-|node258) -- (node258linetop);
\draw [rounded corners=\railcorners] (node252.east) -- (node252.east-|node258) -- (node258linetop);
\draw [rounded corners=\railcorners] (node242.east) -- (node242.east-|node246) -- (node246linetop);
\draw [rounded corners=\railcorners] (node244.east) -- (node244.east-|node246) -- (node246linetop);
\draw [rounded corners=\railcorners] (coord12) -- (coord12-|node252) -- (node252linetop);
\draw [rounded corners=\railcorners] (node255.east) -- (node255.east-|node258) -- (node258linetop);
\draw [rounded corners=\railcorners] (node256.east) -- (node256.east-|node258) -- (node258linetop);
\end{tikzpicture}
}
\caption{No Caption.}
\label{No Caption.}
\end{figure}

\begin{figure}
\centerline{
\begin{tikzpicture}
\node at (0pt,0pt)[anchor=west](name){\railname{hmm\strut}};
\coordinate (node268) at (38pt,-21pt);
\coordinate (node268linetop) at (38pt,-27pt);
\coordinate (node268linebottom) at (38pt,-57pt);
\draw [rounded corners=\railcorners] (node268linetop) -- (node268linebottom);
\draw [rounded corners=\railcorners] (node268linetop) -- (node268) -- +(west:8pt);
\coordinate (node269) at (95.15pt,-21pt);
\coordinate (node269linetop) at (95.15pt,-27pt);
\coordinate (node269linebottom) at (95.15pt,-57pt);
\draw [rounded corners=\railcorners] (node269linetop) -- (node269linebottom);
\draw [rounded corners=\railcorners] (node269linetop) -- (node269) -- +(east:8pt);
\node (node263) at (16pt,-21pt)[anchor=west,nonterminal] {\railname{a\strut}};
\writenodesize{node263}
\draw [rounded corners=\railcorners] (node263.east) -- (node268.west);
\node (node264) at (48.06pt,-21pt)[anchor=west,nonterminal] {\railname{that\_one\strut}};
\writenodesize{node264}
\node (node265) at (46pt,-43pt)[anchor=west,nonterminal] {\railname{other\_one\strut}};
\writenodesize{node265}
\node (node267) at (50.2262pt,-65pt)[anchor=west,terminal] {\railtermname{none\strut}};
\writenodesize{node267}
\draw [rounded corners=\railcorners] (node268.east) -- (node264.west);
\draw [rounded corners=\railcorners] (node264.east) -- (node269.west);
\node (node272) at (103.15pt,-21pt)[anchor=west,nonterminal] {\railname{other\strut}};
\writenodesize{node272}
\draw [rounded corners=\railcorners] (node269.east) -- (node272.west);
\draw [rounded corners=\railcorners] (node265.west) -- (node265.west-|node268) -- (node268linetop);
\draw [rounded corners=\railcorners] (node267.west) -- (node267.west-|node268) -- (node268linetop);
\draw [rounded corners=\railcorners] (node265.east) -- (node265.east-|node269) -- (node269linetop);
\draw [rounded corners=\railcorners] (node267.east) -- (node267.east-|node269) -- (node269linetop);
\end{tikzpicture}
}
\caption{No Caption.}
\label{No Caption.}
\end{figure}

\begin{figure}
\centerline{
\begin{tikzpicture}
\node at (0pt,0pt)[anchor=west](name){\railname{aawesome\_list\strut}};
\coordinate (node308) at (38pt,-21pt);
\coordinate (node308linetop) at (38pt,-27pt);
\coordinate (node308linebottom) at (38pt,-204pt);
\draw [rounded corners=\railcorners] (node308linetop) -- (node308linebottom);
\draw [rounded corners=\railcorners] (node308linetop) -- (node308) -- +(west:8pt);
\coordinate (node287) at (71.5099pt,-65pt);
\coordinate (node287linetop) at (71.5099pt,-71pt);
\coordinate (node287linebottom) at (71.5099pt,-94pt);
\draw [rounded corners=\railcorners] (node287linetop) -- (node287linebottom);
\draw [rounded corners=\railcorners] (node287linetop) -- (node287) -- +(west:8pt);
\coordinate (node309) at (133.228pt,-21pt);
\coordinate (node309linetop) at (133.228pt,-27pt);
\coordinate (node309linebottom) at (133.228pt,-204pt);
\draw [rounded corners=\railcorners] (node309linetop) -- (node309linebottom);
\draw [rounded corners=\railcorners] (node309linetop) -- (node309) -- +(east:8pt);
\node (node274) at (16pt,-21pt)[anchor=west,nonterminal] {\railname{a\strut}};
\writenodesize{node274}
\draw [rounded corners=\railcorners] (node274.east) -- (node308.west);
\node (node275) at (77.6138pt,-21pt)[anchor=west,terminal] {\railtermname{b\strut}};
\writenodesize{node275}
\node (node276) at (78.6138pt,-43pt)[anchor=west,nonterminal] {\railname{c\strut}};
\writenodesize{node276}
\node (node278) at (46pt,-65pt)[anchor=west,nonterminal] {\railname{this\strut}};
\writenodesize{node278}
\draw [rounded corners=\railcorners] (node278.east) -- (node287.west);
\coordinate (coord13) at (102.369pt,-65pt);
\node (node279) at (83.8538pt,-80pt)[anchor=west,nonterminal] {\railname{that\_one\strut}};
\writenodesize{node279}
\node (node280) at (79.5099pt,-102pt)[anchor=west,terminal] {\railtermname{this one\strut}};
\writenodesize{node280}
\draw [rounded corners=\railcorners] (node287.east) -- (coord13);
\node (node291) at (65.0388pt,-124pt)[anchor=west,nonterminal] {\railname{other\_one\strut}};
\writenodesize{node291}
\node (node292) at (72.2849pt,-146pt)[anchor=west,terminal] {\railtermname{and\strut}};
\writenodesize{node292}
\node (node293) at (77.8788pt,-168pt)[anchor=west,nonterminal] {\railname{the\strut}};
\writenodesize{node293}
\node (node295) at (70.4149pt,-190pt)[anchor=west,terminal] {\railtermname{bout\strut}};
\writenodesize{node295}
\node (node304) at (64.41pt,-212pt)[anchor=west,terminal] {\railtermname{neither\strut}};
\writenodesize{node304}
\draw [rounded corners=\railcorners] (node308.east) -- (node275.west);
\draw [rounded corners=\railcorners] (node275.east) -- (node309.west);
\node (node312) at (141.228pt,-21pt)[anchor=west,nonterminal] {\railname{other\strut}};
\writenodesize{node312}
\draw [rounded corners=\railcorners] (node309.east) -- (node312.west);
\draw [rounded corners=\railcorners] (node276.west) -- (node276.west-|node308) -- (node308linetop);
\draw [rounded corners=\railcorners] (node278.west) -- (node278.west-|node308) -- (node308linetop);
\draw [rounded corners=\railcorners] (node278.west) -- (node278.west-|node308) -- (node308linetop);
\draw [rounded corners=\railcorners] (node279.west) -- (node279.west-|node287) -- (node287linetop);
\draw [rounded corners=\railcorners] (node280.west) -- (node280.west-|node287) -- (node287linetop);
\draw [rounded corners=\railcorners] (node291.west) -- (node291.west-|node308) -- (node308linetop);
\draw [rounded corners=\railcorners] (node292.west) -- (node292.west-|node308) -- (node308linetop);
\draw [rounded corners=\railcorners] (node293.west) -- (node293.west-|node308) -- (node308linetop);
\draw [rounded corners=\railcorners] (node295.west) -- (node295.west-|node308) -- (node308linetop);
\draw [rounded corners=\railcorners] (node304.west) -- (node304.west-|node308) -- (node308linetop);
\draw [rounded corners=\railcorners] (node276.east) -- (node276.east-|node309) -- (node309linetop);
\draw [rounded corners=\railcorners] (coord13) -- (coord13-|node309) -- (node309linetop);
\draw [rounded corners=\railcorners] (coord13) -- (coord13-|node309) -- (node309linetop);
\draw [rounded corners=\railcorners] (node279.east) -- (node279.east-|node309) -- (node309linetop);
\draw [rounded corners=\railcorners] (node280.east) -- (node280.east-|node309) -- (node309linetop);
\draw [rounded corners=\railcorners] (node279.east) -- (node279.east-|node309) -- (node309linetop);
\draw [rounded corners=\railcorners] (node280.east) -- (node280.east-|node309) -- (node309linetop);
\draw [rounded corners=\railcorners] (node291.east) -- (node291.east-|node309) -- (node309linetop);
\draw [rounded corners=\railcorners] (node292.east) -- (node292.east-|node309) -- (node309linetop);
\draw [rounded corners=\railcorners] (node293.east) -- (node293.east-|node309) -- (node309linetop);
\draw [rounded corners=\railcorners] (node295.east) -- (node295.east-|node309) -- (node309linetop);
\draw [rounded corners=\railcorners] (node304.east) -- (node304.east-|node309) -- (node309linetop);
\end{tikzpicture}
}
\caption{No Caption.}
\label{No Caption.}
\end{figure}

\begin{figure}
\centerline{
\begin{tikzpicture}
\node at (0pt,0pt)[anchor=west](name){\railname{grammar\strut}};
\coordinate (node318) at (16pt,-21pt);
\coordinate (node318linetop) at (16pt,-27pt);
\coordinate (node318linebottom) at (16pt,-35pt);
\draw [rounded corners=\railcorners] (node318linetop) -- (node318linebottom);
\draw [rounded corners=\railcorners] (node318linetop) -- (node318) -- +(east:8pt);
\coordinate (node319) at (50.84pt,-21pt);
\coordinate (node319linetop) at (50.84pt,-27pt);
\coordinate (node319linebottom) at (50.84pt,-35pt);
\draw [rounded corners=\railcorners] (node319linetop) -- (node319linebottom);
\draw [rounded corners=\railcorners] (node319linetop) -- (node319) -- +(west:8pt);
\node (node314) at (24pt,-21pt)[anchor=west,nonterminal] {\railname{rule\strut}};
\writenodesize{node314}
\coordinate (coord14) at (33.42pt,-43pt);
\draw [rounded corners=\railcorners] (node318.east) -- (node314.west);
\draw [rounded corners=\railcorners] (node314.east) -- (node319.west);
\draw [rounded corners=\railcorners] (coord14) -- (coord14-|node318) -- (node318linetop);
\draw [rounded corners=\railcorners] (coord14) -- (coord14-|node319) -- (node319linetop);
\end{tikzpicture}
}
\caption{No Caption.}
\label{No Caption.}
\end{figure}

\begin{figure}
\centerline{
\begin{tikzpicture}
\node at (0pt,0pt)[anchor=west](name){\railname{other\strut}};
\coordinate (node336) at (60pt,-21pt);
\coordinate (node336linetop) at (60pt,-27pt);
\coordinate (node336linebottom) at (60pt,-28pt);
\draw [rounded corners=\railcorners] (node336linetop) -- (node336linebottom);
\draw [rounded corners=\railcorners] (node336linetop) -- (node336) -- +(west:8pt);
\coordinate (node337) at (244pt,-21pt);
\coordinate (node337linetop) at (244pt,-27pt);
\coordinate (node337linebottom) at (244pt,-28pt);
\draw [rounded corners=\railcorners] (node337linetop) -- (node337linebottom);
\draw [rounded corners=\railcorners] (node337linetop) -- (node337) -- +(east:8pt);
\node (node322) at (16pt,-21pt)[anchor=west,nonterminal] {\railname{a\strut}};
\writenodesize{node322}
\node (node323) at (38pt,-21pt)[anchor=west,nonterminal] {\railname{b\strut}};
\writenodesize{node323}
\draw [rounded corners=\railcorners] (node322.east) -- (node323.west);
\draw [rounded corners=\railcorners] (node323.east) -- (node336.west);
\coordinate (coord15) at (152pt,-21pt);
\node (node325) at (68pt,-36pt)[anchor=west,nonterminal] {\railname{c\strut}};
\writenodesize{node325}
\node (node326) at (90pt,-36pt)[anchor=west,nonterminal] {\railname{d\strut}};
\writenodesize{node326}
\draw [rounded corners=\railcorners] (node325.east) -- (node326.west);
\node (node328) at (112pt,-36pt)[anchor=west,nonterminal] {\railname{e\strut}};
\writenodesize{node328}
\draw [rounded corners=\railcorners] (node326.east) -- (node328.west);
\node (node329) at (134pt,-36pt)[anchor=west,nonterminal] {\railname{f\strut}};
\writenodesize{node329}
\draw [rounded corners=\railcorners] (node328.east) -- (node329.west);
\node (node330) at (156pt,-36pt)[anchor=west,nonterminal] {\railname{g\strut}};
\writenodesize{node330}
\draw [rounded corners=\railcorners] (node329.east) -- (node330.west);
\node (node331) at (178pt,-36pt)[anchor=west,nonterminal] {\railname{h\strut}};
\writenodesize{node331}
\draw [rounded corners=\railcorners] (node330.east) -- (node331.west);
\node (node332) at (200pt,-36pt)[anchor=west,nonterminal] {\railname{i\strut}};
\writenodesize{node332}
\draw [rounded corners=\railcorners] (node331.east) -- (node332.west);
\node (node333) at (222pt,-36pt)[anchor=west,nonterminal] {\railname{j\strut}};
\writenodesize{node333}
\draw [rounded corners=\railcorners] (node332.east) -- (node333.west);
\draw [rounded corners=\railcorners] (node336.east) -- (coord15);
\draw [rounded corners=\railcorners] (coord15) -- (node337.west);
\node (node339) at (252pt,-21pt)[anchor=west,nonterminal] {\railname{k\strut}};
\writenodesize{node339}
\draw [rounded corners=\railcorners] (node337.east) -- (node339.west);
\node (node340) at (274pt,-21pt)[anchor=west,nonterminal] {\railname{l\strut}};
\writenodesize{node340}
\draw [rounded corners=\railcorners] (node339.east) -- (node340.west);
\node (node342) at (296pt,-21pt)[anchor=west,nonterminal] {\railname{m\strut}};
\writenodesize{node342}
\draw [rounded corners=\railcorners] (node340.east) -- (node342.west);
\node (node343) at (318pt,-21pt)[anchor=west,nonterminal] {\railname{n\strut}};
\writenodesize{node343}
\draw [rounded corners=\railcorners] (node342.east) -- (node343.west);
\node (node344) at (340pt,-21pt)[anchor=west,nonterminal] {\railname{o\strut}};
\writenodesize{node344}
\draw [rounded corners=\railcorners] (node343.east) -- (node344.west);
\draw [rounded corners=\railcorners] (node325.west) -- (node325.west-|node336) -- (node336linetop);
\draw [rounded corners=\railcorners] (node325.west) -- (node325.west-|node336) -- (node336linetop);
\draw [rounded corners=\railcorners] (node333.east) -- (node333.east-|node337) -- (node337linetop);
\draw [rounded corners=\railcorners] (node333.east) -- (node333.east-|node337) -- (node337linetop);
\end{tikzpicture}
}
\caption{No Caption.}
\label{No Caption.}
\end{figure}

\begin{figure}
\centerline{
\begin{tikzpicture}
\node at (0pt,0pt)[anchor=west](name){\railname{otherb\strut}};
\node (node347) at (16pt,-21pt)[anchor=west,nonterminal] {\railname{a\strut}};
\writenodesize{node347}
\node (node348) at (38pt,-21pt)[anchor=west,nonterminal] {\railname{b\strut}};
\writenodesize{node348}
\draw [rounded corners=\railcorners] (node347.east) -- (node348.west);
\node (node350) at (60pt,-21pt)[anchor=west,nonterminal] {\railname{c\strut}};
\writenodesize{node350}
\draw [rounded corners=\railcorners] (node348.east) -- (node350.west);
\node (node351) at (82pt,-21pt)[anchor=west,nonterminal] {\railname{d\strut}};
\writenodesize{node351}
\draw [rounded corners=\railcorners] (node350.east) -- (node351.west);
\node (node353) at (104pt,-21pt)[anchor=west,nonterminal] {\railname{e\strut}};
\writenodesize{node353}
\draw [rounded corners=\railcorners] (node351.east) -- (node353.west);
\node (node354) at (126pt,-21pt)[anchor=west,nonterminal] {\railname{f\strut}};
\writenodesize{node354}
\draw [rounded corners=\railcorners] (node353.east) -- (node354.west);
\node (node355) at (148pt,-21pt)[anchor=west,nonterminal] {\railname{g\strut}};
\writenodesize{node355}
\draw [rounded corners=\railcorners] (node354.east) -- (node355.west);
\node (node356) at (170pt,-21pt)[anchor=west,nonterminal] {\railname{h\strut}};
\writenodesize{node356}
\draw [rounded corners=\railcorners] (node355.east) -- (node356.west);
\node (node358) at (192pt,-21pt)[anchor=west,nonterminal] {\railname{i\strut}};
\writenodesize{node358}
\draw [rounded corners=\railcorners] (node356.east) -- (node358.west);
\node (node359) at (214pt,-21pt)[anchor=west,nonterminal] {\railname{j\strut}};
\writenodesize{node359}
\draw [rounded corners=\railcorners] (node358.east) -- (node359.west);
\node (node360) at (236pt,-21pt)[anchor=west,nonterminal] {\railname{k\strut}};
\writenodesize{node360}
\draw [rounded corners=\railcorners] (node359.east) -- (node360.west);
\node (node361) at (258pt,-21pt)[anchor=west,nonterminal] {\railname{l\strut}};
\writenodesize{node361}
\draw [rounded corners=\railcorners] (node360.east) -- (node361.west);
\node (node363) at (280pt,-21pt)[anchor=west,nonterminal] {\railname{m\strut}};
\writenodesize{node363}
\draw [rounded corners=\railcorners] (node361.east) -- (node363.west);
\node (node364) at (302pt,-21pt)[anchor=west,nonterminal] {\railname{n\strut}};
\writenodesize{node364}
\draw [rounded corners=\railcorners] (node363.east) -- (node364.west);
\node (node365) at (324pt,-21pt)[anchor=west,nonterminal] {\railname{o\strut}};
\writenodesize{node365}
\draw [rounded corners=\railcorners] (node364.east) -- (node365.west);
\end{tikzpicture}
}
\caption{No Caption.}
\label{No Caption.}
\end{figure}

\begin{figure}
\centerline{
\begin{tikzpicture}
\node at (0pt,0pt)[anchor=west](name){\railname{easyone\strut}};
\coordinate (node376) at (16pt,-21pt);
\coordinate (node376linetop) at (16pt,-27pt);
\coordinate (node376linebottom) at (16pt,-50pt);
\draw [rounded corners=\railcorners] (node376linetop) -- (node376linebottom);
\draw [rounded corners=\railcorners] (node376linetop) -- (node376) -- +(west:8pt);
\coordinate (node377) at (46pt,-21pt);
\coordinate (node377linetop) at (46pt,-27pt);
\coordinate (node377linebottom) at (46pt,-50pt);
\draw [rounded corners=\railcorners] (node377linetop) -- (node377linebottom);
\draw [rounded corners=\railcorners] (node377linetop) -- (node377) -- +(east:8pt);
\coordinate (coord16) at (31pt,-21pt);
\node (node368) at (24pt,-36pt)[anchor=west,nonterminal] {\railname{a\strut}};
\writenodesize{node368}
\node (node369) at (24pt,-58pt)[anchor=west,nonterminal] {\railname{b\strut}};
\writenodesize{node369}
\draw [rounded corners=\railcorners] (node376.east) -- (coord16);
\draw [rounded corners=\railcorners] (coord16) -- (node377.west);
\draw [rounded corners=\railcorners] (node368.west) -- (node368.west-|node376) -- (node376linetop);
\draw [rounded corners=\railcorners] (node369.west) -- (node369.west-|node376) -- (node376linetop);
\draw [rounded corners=\railcorners] (node368.east) -- (node368.east-|node377) -- (node377linetop);
\draw [rounded corners=\railcorners] (node369.east) -- (node369.east-|node377) -- (node377linetop);
\end{tikzpicture}
}
\caption{No Caption.}
\label{No Caption.}
\end{figure}

\begin{figure}
\centerline{
\begin{tikzpicture}
\node at (0pt,0pt)[anchor=west](name){\railname{a\_better\_loop\strut}};
\coordinate (node390) at (16pt,-21pt);
\coordinate (node390linetop) at (16pt,-27pt);
\coordinate (node390linebottom) at (16pt,-35pt);
\draw [rounded corners=\railcorners] (node390linetop) -- (node390linebottom);
\draw [rounded corners=\railcorners] (node390linetop) -- (node390) -- +(east:8pt);
\coordinate (node391) at (46pt,-21pt);
\coordinate (node391linetop) at (46pt,-27pt);
\coordinate (node391linebottom) at (46pt,-35pt);
\draw [rounded corners=\railcorners] (node391linetop) -- (node391linebottom);
\draw [rounded corners=\railcorners] (node391linetop) -- (node391) -- +(west:8pt);
\node (node381) at (24pt,-21pt)[anchor=west,nonterminal] {\railname{a\strut}};
\writenodesize{node381}
\coordinate (coord17) at (31pt,-43pt);
\draw [rounded corners=\railcorners] (node390.east) -- (node381.west);
\draw [rounded corners=\railcorners] (node381.east) -- (node391.west);
\draw [rounded corners=\railcorners] (coord17) -- (coord17-|node390) -- (node390linetop);
\draw [rounded corners=\railcorners] (coord17) -- (coord17-|node391) -- (node391linetop);
\end{tikzpicture}
}
\caption{No Caption.}
\label{No Caption.}
\end{figure}

\begin{figure}
\centerline{
\begin{tikzpicture}
\node at (0pt,0pt)[anchor=west](name){\railname{a\_better\_loopb\strut}};
\coordinate (node405) at (38pt,-21pt);
\coordinate (node405linetop) at (38pt,-27pt);
\coordinate (node405linebottom) at (38pt,-28pt);
\draw [rounded corners=\railcorners] (node405linetop) -- (node405linebottom);
\draw [rounded corners=\railcorners] (node405linetop) -- (node405) -- +(east:8pt);
\coordinate (node406) at (68pt,-21pt);
\coordinate (node406linetop) at (68pt,-27pt);
\coordinate (node406linebottom) at (68pt,-28pt);
\draw [rounded corners=\railcorners] (node406linetop) -- (node406linebottom);
\draw [rounded corners=\railcorners] (node406linetop) -- (node406) -- +(west:8pt);
\node (node395) at (16pt,-21pt)[anchor=west,nonterminal] {\railname{a\strut}};
\writenodesize{node395}
\draw [rounded corners=\railcorners] (node395.east) -- (node405.west);
\coordinate (coord19) at (53pt,-21pt);
\node (node396) at (46pt,-36pt)[anchor=west,nonterminal] {\railname{b\strut}};
\writenodesize{node396}
\draw [rounded corners=\railcorners] (node405.east) -- (coord19);
\draw [rounded corners=\railcorners] (coord19) -- (node406.west);
\draw [rounded corners=\railcorners] (node396.west) -- (node396.west-|node405) -- (node405linetop);
\draw [rounded corners=\railcorners] (node396.east) -- (node396.east-|node406) -- (node406linetop);
\end{tikzpicture}
}
\caption{No Caption.}
\label{No Caption.}
\end{figure}

\begin{figure}
\centerline{
\begin{tikzpicture}
\node at (0pt,0pt)[anchor=west](name){\railname{a\_much\_better\_loop\strut}};
\coordinate (node422) at (16pt,-21pt);
\coordinate (node422linetop) at (16pt,-27pt);
\coordinate (node422linebottom) at (16pt,-35pt);
\draw [rounded corners=\railcorners] (node422linetop) -- (node422linebottom);
\draw [rounded corners=\railcorners] (node422linetop) -- (node422) -- +(east:8pt);
\coordinate (node423) at (48pt,-21pt);
\coordinate (node423linetop) at (48pt,-27pt);
\coordinate (node423linebottom) at (48pt,-35pt);
\draw [rounded corners=\railcorners] (node423linetop) -- (node423linebottom);
\draw [rounded corners=\railcorners] (node423linetop) -- (node423) -- +(west:8pt);
\node (node412) at (25pt,-21pt)[anchor=west,nonterminal] {\railname{a\strut}};
\writenodesize{node412}
\node (node411) at (24pt,-43pt)[anchor=west,terminal] {\railtermname{,\strut}};
\writenodesize{node411}
\draw [rounded corners=\railcorners] (node422.east) -- (node412.west);
\draw [rounded corners=\railcorners] (node412.east) -- (node423.west);
\draw [rounded corners=\railcorners] (node411.west) -- (node411.west-|node422) -- (node422linetop);
\draw [rounded corners=\railcorners] (node411.east) -- (node411.east-|node423) -- (node423linetop);
\end{tikzpicture}
}
\caption{No Caption.}
\label{No Caption.}
\end{figure}

\begin{figure}
\centerline{
\begin{tikzpicture}
\node at (0pt,0pt)[anchor=west](name){\railname{a\_much\_better\_loop\strut}};
\coordinate (node438) at (38pt,-21pt);
\coordinate (node438linetop) at (38pt,-27pt);
\coordinate (node438linebottom) at (38pt,-35pt);
\draw [rounded corners=\railcorners] (node438linetop) -- (node438linebottom);
\draw [rounded corners=\railcorners] (node438linetop) -- (node438) -- +(east:8pt);
\coordinate (node439) at (90pt,-21pt);
\coordinate (node439linetop) at (90pt,-27pt);
\coordinate (node439linebottom) at (90pt,-35pt);
\draw [rounded corners=\railcorners] (node439linetop) -- (node439linebottom);
\draw [rounded corners=\railcorners] (node439linetop) -- (node439) -- +(west:8pt);
\node (node427) at (16pt,-21pt)[anchor=west,nonterminal] {\railname{x\strut}};
\writenodesize{node427}
\draw [rounded corners=\railcorners] (node427.east) -- (node438.west);
\node (node432) at (46pt,-21pt)[anchor=west,nonterminal] {\railname{a\strut}};
\writenodesize{node432}
\node (node434) at (68pt,-21pt)[anchor=west,nonterminal] {\railname{b\strut}};
\writenodesize{node434}
\draw [rounded corners=\railcorners] (node432.east) -- (node434.west);
\node (node431) at (56pt,-43pt)[anchor=west,terminal] {\railtermname{,\strut}};
\writenodesize{node431}
\draw [rounded corners=\railcorners] (node438.east) -- (node432.west);
\draw [rounded corners=\railcorners] (node434.east) -- (node439.west);
\draw [rounded corners=\railcorners] (node431.west) -- (node431.west-|node438) -- (node438linetop);
\draw [rounded corners=\railcorners] (node431.east) -- (node431.east-|node439) -- (node439linetop);
\end{tikzpicture}
}
\caption{No Caption.}
\label{No Caption.}
\end{figure}

\begin{figure}
\centerline{
\begin{tikzpicture}
\node at (0pt,0pt)[anchor=west](name){\railname{wowzer\strut}};
\coordinate (node461) at (82pt,-21pt);
\coordinate (node461linetop) at (82pt,-27pt);
\coordinate (node461linebottom) at (82pt,-28pt);
\draw [rounded corners=\railcorners] (node461linetop) -- (node461linebottom);
\draw [rounded corners=\railcorners] (node461linetop) -- (node461) -- +(east:8pt);
\coordinate (node462) at (243.688pt,-21pt);
\coordinate (node462linetop) at (243.688pt,-27pt);
\coordinate (node462linebottom) at (243.688pt,-28pt);
\draw [rounded corners=\railcorners] (node462linetop) -- (node462linebottom);
\draw [rounded corners=\railcorners] (node462linetop) -- (node462) -- +(west:8pt);
\node (node442) at (16pt,-21pt)[anchor=west,nonterminal] {\railname{x\strut}};
\writenodesize{node442}
\node (node443) at (38pt,-21pt)[anchor=west,nonterminal] {\railname{a\strut}};
\writenodesize{node443}
\draw [rounded corners=\railcorners] (node442.east) -- (node443.west);
\node (node445) at (60pt,-21pt)[anchor=west,nonterminal] {\railname{b\strut}};
\writenodesize{node445}
\draw [rounded corners=\railcorners] (node443.east) -- (node445.west);
\draw [rounded corners=\railcorners] (node445.east) -- (node461.west);
\coordinate (coord24) at (162.844pt,-21pt);
\node (node452) at (90pt,-36pt)[anchor=west,nonterminal] {\railname{g\strut}};
\writenodesize{node452}
\node (node451) at (112pt,-36pt)[anchor=west,nonterminal] {\railname{b\strut}};
\writenodesize{node451}
\draw [rounded corners=\railcorners] (node452.east) -- (node451.west);
\node (node450) at (134pt,-36pt)[anchor=west,nonterminal] {\railname{a\strut}};
\writenodesize{node450}
\draw [rounded corners=\railcorners] (node451.east) -- (node450.west);
\node (node449) at (156pt,-36pt)[anchor=west,terminal] {\railtermname{part\strut}};
\writenodesize{node449}
\draw [rounded corners=\railcorners] (node450.east) -- (node449.west);
\node (node447) at (191.618pt,-36pt)[anchor=west,terminal] {\railtermname{,\strut}};
\writenodesize{node447}
\draw [rounded corners=\railcorners] (node449.east) -- (node447.west);
\node (node446) at (215.618pt,-36pt)[anchor=west,nonterminal] {\railname{blue\strut}};
\writenodesize{node446}
\draw [rounded corners=\railcorners] (node447.east) -- (node446.west);
\draw [rounded corners=\railcorners] (node461.east) -- (coord24);
\draw [rounded corners=\railcorners] (coord24) -- (node462.west);
\draw [rounded corners=\railcorners] (node452.west) -- (node452.west-|node461) -- (node461linetop);
\draw [rounded corners=\railcorners] (node452.west) -- (node452.west-|node461) -- (node461linetop);
\draw [rounded corners=\railcorners] (node446.east) -- (node446.east-|node462) -- (node462linetop);
\draw [rounded corners=\railcorners] (node446.east) -- (node446.east-|node462) -- (node462linetop);
\end{tikzpicture}
}
\caption{No Caption.}
\label{No Caption.}
\end{figure}

\begin{figure}
\centerline{
\begin{tikzpicture}
\node at (0pt,0pt)[anchor=west](name){\railname{gleep\strut}};
\coordinate (node481) at (38pt,-21pt);
\coordinate (node481linetop) at (38pt,-27pt);
\coordinate (node481linebottom) at (38pt,-35pt);
\draw [rounded corners=\railcorners] (node481linetop) -- (node481linebottom);
\draw [rounded corners=\railcorners] (node481linetop) -- (node481) -- +(east:8pt);
\coordinate (node482) at (90pt,-21pt);
\coordinate (node482linetop) at (90pt,-27pt);
\coordinate (node482linebottom) at (90pt,-35pt);
\draw [rounded corners=\railcorners] (node482linetop) -- (node482linebottom);
\draw [rounded corners=\railcorners] (node482linetop) -- (node482) -- +(west:8pt);
\node (node465) at (16pt,-21pt)[anchor=west,nonterminal] {\railname{x\strut}};
\writenodesize{node465}
\draw [rounded corners=\railcorners] (node465.east) -- (node481.west);
\node (node470) at (46pt,-21pt)[anchor=west,nonterminal] {\railname{a\strut}};
\writenodesize{node470}
\node (node472) at (68pt,-21pt)[anchor=west,nonterminal] {\railname{b\strut}};
\writenodesize{node472}
\draw [rounded corners=\railcorners] (node470.east) -- (node472.west);
\node (node469) at (56pt,-43pt)[anchor=west,terminal] {\railtermname{,\strut}};
\writenodesize{node469}
\draw [rounded corners=\railcorners] (node481.east) -- (node470.west);
\draw [rounded corners=\railcorners] (node472.east) -- (node482.west);
\node (node484) at (98pt,-21pt)[anchor=west,nonterminal] {\railname{right\strut}};
\writenodesize{node484}
\draw [rounded corners=\railcorners] (node482.east) -- (node484.west);
\draw [rounded corners=\railcorners] (node469.west) -- (node469.west-|node481) -- (node481linetop);
\draw [rounded corners=\railcorners] (node469.east) -- (node469.east-|node482) -- (node482linetop);
\end{tikzpicture}
}
\caption{No Caption.}
\label{No Caption.}
\end{figure}

\begin{figure}
\centerline{
\begin{tikzpicture}
\node at (0pt,0pt)[anchor=west](name){\railname{muckrake\strut}};
\coordinate (node503) at (38pt,-21pt);
\coordinate (node503linetop) at (38pt,-27pt);
\coordinate (node503linebottom) at (38pt,-35pt);
\draw [rounded corners=\railcorners] (node503linetop) -- (node503linebottom);
\draw [rounded corners=\railcorners] (node503linetop) -- (node503) -- +(east:8pt);
\coordinate (node504) at (98.07pt,-21pt);
\coordinate (node504linetop) at (98.07pt,-27pt);
\coordinate (node504linebottom) at (98.07pt,-35pt);
\draw [rounded corners=\railcorners] (node504linetop) -- (node504linebottom);
\draw [rounded corners=\railcorners] (node504linetop) -- (node504) -- +(west:8pt);
\node (node486) at (16pt,-21pt)[anchor=west,nonterminal] {\railname{x\strut}};
\writenodesize{node486}
\draw [rounded corners=\railcorners] (node486.east) -- (node503.west);
\node (node493) at (50.035pt,-21pt)[anchor=west,nonterminal] {\railname{a\strut}};
\writenodesize{node493}
\node (node494) at (72.035pt,-21pt)[anchor=west,nonterminal] {\railname{b\strut}};
\writenodesize{node494}
\draw [rounded corners=\railcorners] (node493.east) -- (node494.west);
\node (node491) at (46pt,-43pt)[anchor=west,terminal] {\railtermname{,\strut}};
\writenodesize{node491}
\node (node490) at (70pt,-43pt)[anchor=west,nonterminal] {\railname{blue\strut}};
\writenodesize{node490}
\draw [rounded corners=\railcorners] (node491.east) -- (node490.west);
\draw [rounded corners=\railcorners] (node503.east) -- (node493.west);
\draw [rounded corners=\railcorners] (node494.east) -- (node504.west);
\node (node506) at (106.07pt,-21pt)[anchor=west,terminal] {\railtermname{toodles\strut}};
\writenodesize{node506}
\draw [rounded corners=\railcorners] (node504.east) -- (node506.west);
\draw [rounded corners=\railcorners] (node491.west) -- (node491.west-|node503) -- (node503linetop);
\draw [rounded corners=\railcorners] (node491.west) -- (node491.west-|node503) -- (node503linetop);
\draw [rounded corners=\railcorners] (node490.east) -- (node490.east-|node504) -- (node504linetop);
\draw [rounded corners=\railcorners] (node490.east) -- (node490.east-|node504) -- (node504linetop);
\end{tikzpicture}
}
\caption{No Caption.}
\label{No Caption.}
\end{figure}

\begin{figure}
\centerline{
\begin{tikzpicture}
\node at (0pt,0pt)[anchor=west](name){\railname{broken\strut}};
\coordinate (node521) at (82pt,-21pt);
\coordinate (node521linetop) at (82pt,-27pt);
\coordinate (node521linebottom) at (82pt,-28pt);
\draw [rounded corners=\railcorners] (node521linetop) -- (node521linebottom);
\draw [rounded corners=\railcorners] (node521linetop) -- (node521) -- +(east:8pt);
\coordinate (node522) at (112pt,-21pt);
\coordinate (node522linetop) at (112pt,-27pt);
\coordinate (node522linebottom) at (112pt,-28pt);
\draw [rounded corners=\railcorners] (node522linetop) -- (node522linebottom);
\draw [rounded corners=\railcorners] (node522linetop) -- (node522) -- +(west:8pt);
\node (node508) at (16pt,-21pt)[anchor=west,nonterminal] {\railname{x\strut}};
\writenodesize{node508}
\node (node509) at (38pt,-21pt)[anchor=west,nonterminal] {\railname{a\strut}};
\writenodesize{node509}
\draw [rounded corners=\railcorners] (node508.east) -- (node509.west);
\node (node511) at (60pt,-21pt)[anchor=west,nonterminal] {\railname{b\strut}};
\writenodesize{node511}
\draw [rounded corners=\railcorners] (node509.east) -- (node511.west);
\draw [rounded corners=\railcorners] (node511.east) -- (node521.west);
\coordinate (coord30) at (97pt,-21pt);
\node (node512) at (90pt,-36pt)[anchor=west,nonterminal] {\railname{x\strut}};
\writenodesize{node512}
\draw [rounded corners=\railcorners] (node521.east) -- (coord30);
\draw [rounded corners=\railcorners] (coord30) -- (node522.west);
\draw [rounded corners=\railcorners] (node512.west) -- (node512.west-|node521) -- (node521linetop);
\draw [rounded corners=\railcorners] (node512.east) -- (node512.east-|node522) -- (node522linetop);
\end{tikzpicture}
}
\caption{No Caption.}
\label{No Caption.}
\end{figure}

\begin{figure}
\centerline{
\begin{tikzpicture}
\node at (0pt,0pt)[anchor=west](name){\railname{scooby\strut}};
\coordinate (node544) at (16pt,-21pt);
\coordinate (node544linetop) at (16pt,-27pt);
\coordinate (node544linebottom) at (16pt,-28pt);
\draw [rounded corners=\railcorners] (node544linetop) -- (node544linebottom);
\draw [rounded corners=\railcorners] (node544linetop) -- (node544) -- +(east:8pt);
\coordinate (node533) at (24pt,-36pt);
\coordinate (node533linetop) at (24pt,-42pt);
\coordinate (node533linebottom) at (24pt,-50pt);
\draw [rounded corners=\railcorners] (node533linetop) -- (node533linebottom);
\draw [rounded corners=\railcorners] (node533linetop) -- (node533) -- +(west:8pt);
\coordinate (node534) at (54pt,-36pt);
\coordinate (node534linetop) at (54pt,-42pt);
\coordinate (node534linebottom) at (54pt,-50pt);
\draw [rounded corners=\railcorners] (node534linetop) -- (node534linebottom);
\draw [rounded corners=\railcorners] (node534linetop) -- (node534) -- +(east:8pt);
\coordinate (node545) at (171.688pt,-21pt);
\coordinate (node545linetop) at (171.688pt,-27pt);
\coordinate (node545linebottom) at (171.688pt,-28pt);
\draw [rounded corners=\railcorners] (node545linetop) -- (node545linebottom);
\draw [rounded corners=\railcorners] (node545linetop) -- (node545) -- +(west:8pt);
\coordinate (coord32) at (93.8438pt,-21pt);
\node (node530) at (32pt,-36pt)[anchor=west,nonterminal] {\railname{b\strut}};
\writenodesize{node530}
\node (node531) at (32pt,-58pt)[anchor=west,nonterminal] {\railname{g\strut}};
\writenodesize{node531}
\draw [rounded corners=\railcorners] (node533.east) -- (node530.west);
\draw [rounded corners=\railcorners] (node530.east) -- (node534.west);
\node (node529) at (62pt,-36pt)[anchor=west,nonterminal] {\railname{a\strut}};
\writenodesize{node529}
\draw [rounded corners=\railcorners] (node534.east) -- (node529.west);
\node (node528) at (84pt,-36pt)[anchor=west,terminal] {\railtermname{part\strut}};
\writenodesize{node528}
\draw [rounded corners=\railcorners] (node529.east) -- (node528.west);
\node (node526) at (119.618pt,-36pt)[anchor=west,terminal] {\railtermname{,\strut}};
\writenodesize{node526}
\draw [rounded corners=\railcorners] (node528.east) -- (node526.west);
\node (node525) at (143.618pt,-36pt)[anchor=west,nonterminal] {\railname{blue\strut}};
\writenodesize{node525}
\draw [rounded corners=\railcorners] (node526.east) -- (node525.west);
\draw [rounded corners=\railcorners] (node544.east) -- (coord32);
\draw [rounded corners=\railcorners] (coord32) -- (node545.west);
\draw [rounded corners=\railcorners] (node533.west) -- (node533.west-|node544) -- (node544linetop);
\draw [rounded corners=\railcorners] (node531.west) -- (node531.west-|node533) -- (node533linetop);
\draw [rounded corners=\railcorners] (node525.east) -- (node525.east-|node545) -- (node545linetop);
\draw [rounded corners=\railcorners] (node525.east) -- (node525.east-|node545) -- (node545linetop);
\draw [rounded corners=\railcorners] (node531.east) -- (node531.east-|node534) -- (node534linetop);
\end{tikzpicture}
}
\caption{No Caption.}
\label{No Caption.}
\end{figure}

\begin{figure}
\centerline{
\begin{tikzpicture}
\node at (0pt,0pt)[anchor=west](name){\railname{scooby\strut}};
\coordinate (node571) at (82pt,-21pt);
\coordinate (node571linetop) at (82pt,-27pt);
\coordinate (node571linebottom) at (82pt,-28pt);
\draw [rounded corners=\railcorners] (node571linetop) -- (node571linebottom);
\draw [rounded corners=\railcorners] (node571linetop) -- (node571) -- +(east:8pt);
\coordinate (node560) at (90pt,-36pt);
\coordinate (node560linetop) at (90pt,-42pt);
\coordinate (node560linebottom) at (90pt,-50pt);
\draw [rounded corners=\railcorners] (node560linetop) -- (node560linebottom);
\draw [rounded corners=\railcorners] (node560linetop) -- (node560) -- +(west:8pt);
\coordinate (node561) at (120pt,-36pt);
\coordinate (node561linetop) at (120pt,-42pt);
\coordinate (node561linebottom) at (120pt,-50pt);
\draw [rounded corners=\railcorners] (node561linetop) -- (node561linebottom);
\draw [rounded corners=\railcorners] (node561linetop) -- (node561) -- +(east:8pt);
\coordinate (node572) at (237.688pt,-21pt);
\coordinate (node572linetop) at (237.688pt,-27pt);
\coordinate (node572linebottom) at (237.688pt,-28pt);
\draw [rounded corners=\railcorners] (node572linetop) -- (node572linebottom);
\draw [rounded corners=\railcorners] (node572linetop) -- (node572) -- +(west:8pt);
\node (node548) at (16pt,-21pt)[anchor=west,nonterminal] {\railname{x\strut}};
\writenodesize{node548}
\node (node549) at (38pt,-21pt)[anchor=west,nonterminal] {\railname{a\strut}};
\writenodesize{node549}
\draw [rounded corners=\railcorners] (node548.east) -- (node549.west);
\node (node551) at (60pt,-21pt)[anchor=west,nonterminal] {\railname{b\strut}};
\writenodesize{node551}
\draw [rounded corners=\railcorners] (node549.east) -- (node551.west);
\draw [rounded corners=\railcorners] (node551.east) -- (node571.west);
\coordinate (coord34) at (159.844pt,-21pt);
\node (node557) at (98pt,-36pt)[anchor=west,nonterminal] {\railname{b\strut}};
\writenodesize{node557}
\node (node558) at (98pt,-58pt)[anchor=west,nonterminal] {\railname{g\strut}};
\writenodesize{node558}
\draw [rounded corners=\railcorners] (node560.east) -- (node557.west);
\draw [rounded corners=\railcorners] (node557.east) -- (node561.west);
\node (node556) at (128pt,-36pt)[anchor=west,nonterminal] {\railname{a\strut}};
\writenodesize{node556}
\draw [rounded corners=\railcorners] (node561.east) -- (node556.west);
\node (node555) at (150pt,-36pt)[anchor=west,terminal] {\railtermname{part\strut}};
\writenodesize{node555}
\draw [rounded corners=\railcorners] (node556.east) -- (node555.west);
\node (node553) at (185.618pt,-36pt)[anchor=west,terminal] {\railtermname{,\strut}};
\writenodesize{node553}
\draw [rounded corners=\railcorners] (node555.east) -- (node553.west);
\node (node552) at (209.618pt,-36pt)[anchor=west,nonterminal] {\railname{blue\strut}};
\writenodesize{node552}
\draw [rounded corners=\railcorners] (node553.east) -- (node552.west);
\draw [rounded corners=\railcorners] (node571.east) -- (coord34);
\draw [rounded corners=\railcorners] (coord34) -- (node572.west);
\draw [rounded corners=\railcorners] (node560.west) -- (node560.west-|node571) -- (node571linetop);
\draw [rounded corners=\railcorners] (node558.west) -- (node558.west-|node560) -- (node560linetop);
\draw [rounded corners=\railcorners] (node552.east) -- (node552.east-|node572) -- (node572linetop);
\draw [rounded corners=\railcorners] (node552.east) -- (node552.east-|node572) -- (node572linetop);
\draw [rounded corners=\railcorners] (node558.east) -- (node558.east-|node561) -- (node561linetop);
\end{tikzpicture}
}
\caption{No Caption.}
\label{No Caption.}
\end{figure}

\begin{figure}
\centerline{
\begin{tikzpicture}
\node at (0pt,0pt)[anchor=west](name){\railname{muckrake\strut}};
\coordinate (node600) at (38pt,-21pt);
\coordinate (node600linetop) at (38pt,-27pt);
\coordinate (node600linebottom) at (38pt,-57pt);
\draw [rounded corners=\railcorners] (node600linetop) -- (node600linebottom);
\draw [rounded corners=\railcorners] (node600linetop) -- (node600) -- +(east:8pt);
\coordinate (node589) at (53.035pt,-21pt);
\coordinate (node589linetop) at (53.035pt,-27pt);
\coordinate (node589linebottom) at (53.035pt,-35pt);
\draw [rounded corners=\railcorners] (node589linetop) -- (node589linebottom);
\draw [rounded corners=\railcorners] (node589linetop) -- (node589) -- +(west:8pt);
\coordinate (node590) at (83.035pt,-21pt);
\coordinate (node590linetop) at (83.035pt,-27pt);
\coordinate (node590linebottom) at (83.035pt,-35pt);
\draw [rounded corners=\railcorners] (node590linetop) -- (node590linebottom);
\draw [rounded corners=\railcorners] (node590linetop) -- (node590) -- +(east:8pt);
\coordinate (node601) at (98.07pt,-21pt);
\coordinate (node601linetop) at (98.07pt,-27pt);
\coordinate (node601linebottom) at (98.07pt,-57pt);
\draw [rounded corners=\railcorners] (node601linetop) -- (node601linebottom);
\draw [rounded corners=\railcorners] (node601linetop) -- (node601) -- +(west:8pt);
\node (node575) at (16pt,-21pt)[anchor=west,nonterminal] {\railname{x\strut}};
\writenodesize{node575}
\draw [rounded corners=\railcorners] (node575.east) -- (node600.west);
\node (node586) at (61.035pt,-21pt)[anchor=west,nonterminal] {\railname{a\strut}};
\writenodesize{node586}
\node (node587) at (61.035pt,-43pt)[anchor=west,nonterminal] {\railname{b\strut}};
\writenodesize{node587}
\draw [rounded corners=\railcorners] (node589.east) -- (node586.west);
\draw [rounded corners=\railcorners] (node586.east) -- (node590.west);
\node (node584) at (46pt,-65pt)[anchor=west,terminal] {\railtermname{,\strut}};
\writenodesize{node584}
\node (node583) at (70pt,-65pt)[anchor=west,nonterminal] {\railname{blue\strut}};
\writenodesize{node583}
\draw [rounded corners=\railcorners] (node584.east) -- (node583.west);
\draw [rounded corners=\railcorners] (node600.east) -- (node589.west);
\draw [rounded corners=\railcorners] (node590.east) -- (node601.west);
\node (node603) at (106.07pt,-21pt)[anchor=west,terminal] {\railtermname{toodles\strut}};
\writenodesize{node603}
\draw [rounded corners=\railcorners] (node601.east) -- (node603.west);
\draw [rounded corners=\railcorners] (node587.west) -- (node587.west-|node589) -- (node589linetop);
\draw [rounded corners=\railcorners] (node584.west) -- (node584.west-|node600) -- (node600linetop);
\draw [rounded corners=\railcorners] (node584.west) -- (node584.west-|node600) -- (node600linetop);
\draw [rounded corners=\railcorners] (node587.east) -- (node587.east-|node590) -- (node590linetop);
\draw [rounded corners=\railcorners] (node583.east) -- (node583.east-|node601) -- (node601linetop);
\draw [rounded corners=\railcorners] (node583.east) -- (node583.east-|node601) -- (node601linetop);
\end{tikzpicture}
}
\caption{No Caption.}
\label{No Caption.}
\end{figure}


\begin{figure}
  \centerline{\usebox{\catBox}}
  \caption{cat}
  \label{fig:cat}
\end{figure}

\begin{figure}
  \centerline{\usebox{\hatBox}}
  \caption{hat}
  \label{fig:hat}
\end{figure}

\begin{figure}
  \centerline{\usebox{\concatBox}}
  \caption{concat}
  \label{fig:concat}
\end{figure}

\begin{figure}
  \centerline{\usebox{\optionalBox}}
  \caption{optional}
  \label{fig:optional}
\end{figure}

\begin{figure}
  \centerline{\usebox{\concatoptionalBox}}
  \caption{concatoptional}
  \label{fig:concatoptional}
\end{figure}

\begin{figure}
  \centerline{\usebox{\choicetwoBox}}
  \caption{choicetwo}
  \label{fig:choicetwo}
\end{figure}

\begin{figure}
  \centerline{\usebox{\choiceBox}}
  \caption{choice}
  \label{fig:choice}
\end{figure}


\begin{figure}
  \centerline{\usebox{\concatchoiceBox}}
  \caption{concatchoice}
  \label{fig:concatchoice}
\end{figure}


\begin{figure}
  \centerline{\usebox{\optionalchoiceBox}}
  \caption{optionalchoice}
  \label{fig:optionalchoice}
\end{figure}

\begin{figure}
  \centerline{\usebox{\optionalconcatchoiceBox}}
  \caption{optionalconcatchoice}
  \label{fig:optionalconcatchoice}
\end{figure}


\begin{figure}
  \centerline{\usebox{\loopBox}}
  \caption{loop}
  \label{fig:loop}
\end{figure}


\begin{figure}
  \centerline{\usebox{\concatloopBox}}
  \caption{concatloop}
  \label{fig:concatloop}
\end{figure}


\begin{figure}
  \centerline{\usebox{\choiceloopBox}}
  \caption{choiceloop}
  \label{fig:choiceloop}
\end{figure}

\begin{figure}
  \centerline{\usebox{\choiceconcatloopBox}}
  \caption{choiceconcatloop}
  \label{fig:choiceconcatloop}
\end{figure}

\begin{figure}
  \centerline{\usebox{\choiceconcatchoiceloopBox}}
  \caption{choiceconcatchoiceloop}
  \label{fig:choiceconcatchoiceloop}
\end{figure}

.
%% \begin{figure}
%%   \centerline{\usebox{\toollist}}
%%   \caption{Tools.}
%%   \label{fig:tools}
%% \end{figure}
    
%% \begin{figure}
%%   \centerline{\usebox{\otherb}}
%%   \caption{otherb.}
%%   \label{fig:otherb}
%% \end{figure}
    

\immediate\closeout\tempfile

\end{document}
